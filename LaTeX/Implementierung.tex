\chapter{Implementierung} \label{ch:prog}
Der Programmierteil dieser Bachelorarbeit implementiert den Netzwerk-Simplex-Algorithmus wie in \cref{ch:NSA} beschrieben, wobei einzelne Strukturen verändert wurden. Der Code ist in \cpp  geschrieben und hält sich an den \cppelf-Standard. Es folgt eine vollständige Übersicht der Codebestandteile, für das Gesamtkonstrukt siehe TODO Anhang. An einigen Stellen wurden für ein besseres Layout Variablennamen angepasst, Kommentare gekürzt und Funktionsdeklarationen weggelassen.

\section{Netzwerke}
Zur Umsetzung der Graphenstruktur habe ich eine Klasse \lstinline|class Network| geschrieben, die auf \lstinline|struct Node| sowie \lstinline|class Edge| basiert und den Fluss händelt. Damit kann \lstinline|Network| beispielsweise sicherstellen, dass nie Kanten entfernt werden, die aktuell durchflossen sind.

\subsection{Knoten}
\lstinline|struct Node| hat außer einem Konstruktor keinerlei Funktionen, weswegen es nicht in einer eigenen Klasse ausgelagert, sondern in \lstinline|network.h| enthalten ist. 

\begin{lstlisting}
struct Node {
  size_t id;
  intmax_t b_value;
  //defined by (the other) nodeID
  //always incoming and outgoing due to residual edges
  std::set<size_t> neighbours;

  Node (size_t _id, intmax_t _b_value)
    : id(_id), b_value(_b_value) {};
};
\end{lstlisting}

Ein Knoten wird eindeutig durch seine \lstinline|size_t id| identifiziert, weitere Eigenschaften sind sein b-Wert aus $\mathbb{Z}$ und seine Nachbarknoten. Letztere Menge differenziert nicht zwischen Nachbarn durch eingehende und ausgehende Kanten, da diese Unterscheidung im Residualgraph entfällt.

\subsection{Kanten}
Die Klasse \lstinline|class Edge| enthält alle Informationen über Kanten: Kosten, Kapazität, Fluss und die jeweiligen Endknoten. Residualkanten werden durch das Flag \lstinline|isResidual| gekennzeichnet. An dieser Stelle fällt auf, dass die Kapazität nicht nachträglich verändert werden kann, obwohl dies für Residualkanten notwendig wäre. Ich habe mich in meiner Implementierung dafür entschieden, auf Residualkanten Fluss zu erlauben und damit eine veränderliche Kapazität zu simulieren. Dies wird in \cref{ch:graph} näher erläutert.

\begin{lstlisting}
class Edge {
public:
  //edges are initialized with a flow of zero
  Edge (size_t node0, size_t node1, intmax_t cost,
        intmax_t capacity, bool isResidual = false);

  //returns true when flow change was successful
  bool changeFlow (intmax_t value);
  bool changeFlowPossible (intmax_t value);
  //toggles cost betweeen itself and 0
  void toggleCost ();

private:
  size_t node0, node1;
  intmax_t cost, capacity, flow, toggledCost;
  bool isResidual, isToggled = false;
};
\end{lstlisting}

Die Funktion \lstinline|changeFlow| stellt über die Hilfsfunktion \lstinline|changeFlowPossible| sicher, dass bei einer Flussveränderung keine Kapazitätsschranken verletzt werden. Mittels \lstinline|toggleCost| können die Kosten einer Kante zwischen Null und ihrem eigentlichen Wert variiert werden. Somit lässt sich leicht die in \cref{ch:init} beschriebene Low-Cost-Initialisierung umsetzen, siehe dazu \cref{ch:HCLC}.

\subsection{Graph}\label{ch:graph}
Werfen wir einen Blick darauf, wie die Knoten und Kanten in einem Netzwerk gespeichert werden. Ich habe mich dafür entschieden, die Objekte für eine logarithmische Laufzeit von \lstinline|find|, \lstinline|insert| sowie \lstinline|delete| in einer \lstinline|std::map| abzulegen. Die Knoten werden durch ihre \lstinline|size_t id| identifiziert, Kanten wiederum durch Anfangsknoten, Endknoten und die \lstinline|bool isResidual|. Eine \lstinline|std::map| benötigt einen Vergleichsoperator, daher habe ich für die Kanten \lstinline|custComp| definiert, der nach \lstinline|size_t node0|, \lstinline|size_t node1| und zuletzt \lstinline|bool isResidual| vergleicht.

\begin{lstlisting}
class Network {
private:
  intmax_t flow = 0, cost = 0;
  std::vector<size_t> sources, sinks, transit;

  std::map<std::tuple<size_t, size_t, bool>,
           Edge, custComp> edges;
  std::map<size_t, Node, std::less<size_t>> nodes;
};
\end{lstlisting}

Ein Netzwerk unterteilt zusätzlich seine Knoten nach ihrem b-Wert abzüglich des aktuellen Flusses in Quellen, Senken und Transitknoten. Eine Instanz mit einem maximalen Fluss besteht dementsprechend nur aus Transitknoten. Es kennt außerdem seinen \lstinline|intmax_t flow|, also die Anzahl an Fluss, die von den Quellen zu den Senken transportiert wird, und die dafür anfallenden Kosten \lstinline|intmax_t cost|. Für den Algorithmus ist ersteres nicht relevant, da wir nur auf maximalen Flüssen arbeiten.

Der Graph $G$ mit einem Fluss $f$ und der dazugehörige Residualgraph $\bar{G}$ werden nicht als zwei verschiedene Objekte der Netzwerk-Klasse geführt, sondern in einer Instanz gemeinsam gespeichert. Um dies zu bewerkstelligen, wird bei jeder Veränderung an einer Kante gleichzeitig die eindeutige Residualkante aktualisiert. Zur Veranschaulichung folgt der Quellcode von \lstinline|Network::addEdge| ohne die Prüfung der Zulässigkeit.

\begin{lstlisting}
bool Network::addEdge(Edge e) {
  std::tuple<size_t, size_t, bool> key
    = std::make_tuple(e.node0, e.node1, e.isResidual);

  //insert both the edge and the residual edge
  edges.insert(std::make_pair(key, e));
  e.invert();
  edges.insert(std::make_pair(invertKey(key), e));

  nodes.find(node0)->second.neighbours.insert(node1);
  nodes.find(node1)->second.neighbours.insert(node0);
  return true;
}
\end{lstlisting}

Dabei vertauscht \lstinline|invertKey| die beiden Knoten und negiert \lstinline|isResidual|. Die oben nicht erwähnte Funktion \lstinline|Edge::invert| spiegelt die Veränderungen von \lstinline|invertKey| und negiert die Kosten gemäß \cref{defRes}. Die Kapazität wird beibehalten, stattdessen wird der Fluss auf \lstinline|flow = capacity - flow;| gesetzt. Der größte Wert, um den die Residualkante $\bar{e}$ augmentiert werden kann, ist also gleich $f(e)$. Sofern wir $f$ auf Nicht-Residualkanten einschränken, ist diese Implementierung äquivalent zur Definition. Dieser alternative Ansatz fordert noch ein wenig Aufmerksamkeit bei der Klassenfunktion \lstinline|Network::changeFlow|,\footnotemark{} danach können wir bei der Umsetzung des Algorithmus größtenteils ignorieren, ob eine Kante residual ist.

\footnotetext{Für alle Kanten $e$ des Kreises wenden wir \lstinline|e.changeFlow(f)| sowie \lstinline|e.invert().changeFlow(-f)| an und aktualisieren die Kosten auf \lstinline|this->cost += e.cost*f;|.}

Das Netzwerk wird leer oder mit einer Anzahl \lstinline|size_t noOfNodes| von Transitknoten initialisiert und durch Hinzufügen von Knoten und Kanten vervollständigt. Sofern keine Knoten gelöscht werden, ist die \lstinline|size_t id| bei null beginnend lückenlos aufsteigend.

\begin{lstlisting}
class Network {
public:
  Network(size_t noOfNodes);

  bool addEdge(Edge e);
  //returns nodeID
  size_t addNode(intmax_t b_value = 0);
  
  //fails and returns 0 if there's flow left
  bool deleteEdge(size_t node0, size_t node1);
  //fails if there's flow on an edge to this node left
  bool deleteNode(size_t nodeID);

  //takes a path from a source to a sink
  //doesn't use residual edges
  bool addFlow(std::vector<size_t>& path, intmax_t flow);
  bool changeFlow(Circle& c, intmax_t flow);
};
\end{lstlisting}

Die Funktionen \lstinline|deleteEdge| und \lstinline|deleteNode| schlagen fehl, wenn auf den zu löschenden Kanten noch Fluss ist. So kann unter am Ende des Algorithmus durch Löschen des künstlichen Knotens überprüft werden, ob die Instanz lösbar ist. Für Phase 1 des Algorithmus (siehe \cref{ch:HCLC}) nutzen wir die Funktion \lstinline|addFlow|, die einen Weg von einer Quelle zu einer Senke verlangt.

\lstinline|changeFlow| wird in jeder Iteration der zweiten Phase des Netzwerk-Simplex-Algorithmus aufgerufen und augmentiert den übergebenen Kreis \lstinline|Circle c| um einen beliebigen Wert, sofern dies möglich ist. Dieser Wert wird analog zu $\delta$ aus \cref{ch:NSA} der größte zulässige sein. Die hier genutzte \lstinline|class Circle| werden wir uns nun genauer anschauen.

\section{Die Klasse \emph{Circle}}
Sei $G$ ein gerichteter Graph mit $n$ Knoten sowie $m$ Kanten und $T$ ein aufspannender Baum von $G$. In der Herleitung des Algorithmus haben wir genutzt, dass für eine Kante $e\in E(G)\backslash E(T)$ außerhalb des Baumes der Kreis $C_{T,e}$ eindeutig ist. Wird $T$ im Speicher gehalten, kann beispielsweise durch Tiefensuche in Laufzeit $\bigO(n)$ oder mit einer angepassten Datenstruktur (siehe TODO) noch schneller ein Kreis $C_{T,e}$ bzw. dessen Orientierung $\bar{C}_{T,e}$ gefunden werden.

Im Normalfall verändern sich von einer Iteration auf die nächste nur ein Bruchteil der Kreise. Diese lassen sich jedoch schlecht herausfiltern, wenn nur die Veränderung der zulässigen Baumlösung betrachtet wird. Um die Kreise nicht jede Iteration neu berechnen zu müssen, habe ich mich entschieden, nach \cite{betreuer} nicht den Baum abzuspeichern, sondern alle Kreise. Die theoretische Komplexität bleibt etwa dieselbe,\footnotemark{} in der Praxis ist es schneller, nur die veränderten Kreise neu zu berechnen.

\footnotetext{Wenn wir Pivotalgorithmen betrachten, die in Laufzeit $\text{poly}(m-n)$ sämtliche Kreise in ihre Entscheidung einbeziehen, erhalten wir eine Komplexität von $\bigO(n\cdot(m-n) + \text{poly}(m-n))$.}

Die Klasse \lstinline|class Circle| speichert gerichtete Kreise als zwei Vektoren, die Kanten wie gehabt über Anfangsknoten, Endknoten und die \lstinline|bool isResidual| identifizieren. Außerdem speichern sie ihre jeweiligen Werte für $\delta$ in \lstinline|flow| und $c(\bar{C}_{T,e})$ in \lstinline|costPerFlow|; diese müssen jedoch vom Algorithmus gesetzt werden. Nicht explizit in der Klassendeklaration schlägt sich die Invariante nieder, dass im Verlauf des Algorithmus jeder \lstinline|Circle c| durch die erste Kante als einzige Nichtbaumkante spezifiziert wird.

\begin{lstlisting}[escapechar=ß]
//first edge of edges is not part of the underlying tree
class Circle {
public:
  //circles don't know about their graphs
  intmax_t flow = 0, costPerFlow = 0;

private:
  std::vector<std::pair<size_t, size_t>> edges;
  //std::vector<char>, because std::vector<bool> is brokenß\footnotemarkß
  std::vector<char> isResidual;
};
\end{lstlisting}

Bevor wir uns die Umsetzung anschauen, überlegen wir uns theoretisch, was die Klasse leisten muss. Dazu betrachten wir eine Iteration, in der $C_1=\bar{C}_{T,e_1}$ ausgewählt wurde. Sei $C_2=\bar{C}_{T,e_2}$ mit $e_1\neq e_2$ ein weiterer Kreis. \cref{fig:circles} stellt für die zugrundeliegenden ungerichteten Kreise $\tilde{C}_1$ und $\tilde{C}_2$ dar, wie sich $C_1$ und $C_2$ zueinander verhalten können.

\footnotetext{Für mehr Informationen siehe \cite[\textit{Item 18: Avoid using} \texttt{vector<bool>}]{STL}.}

\begin{figure}[!ht]\centering
    \includestandalone{tikz_circlesupdate}
    \caption{Beide Beispiele zeigen je zwei Kreise $\tilde{C}_i=C_{T,e_i}$. Die gestrichelten Linien gehören zum nicht vollständig dargestellten Baum $T$. Im zweiten Fall sind $\tilde{C}_1$ und $\tilde{C}_2$ nicht disjunkt.}
    \label{fig:circles}
\end{figure}

Der Algorithmus wird nun $e_1$ zum neuen Baum $T'$ hinzufügen und eine ausgehende Kante $e_1\neq e'\in E(C_{T,e_1})$ wählen, die dafür entfernt wird. Zunächst halten wir fest, dass es vor der Iteration keinen weiteren Fall als die beiden aus \cref{fig:circles} geben kann: Wären $e_1$ und $e_2$ im Schnitt der ungerichteten Kreise $E(C_{T,e_1})\cap E(C_{T,e_2})$ enthalten, dann wäre $T$ nicht kreisfrei.

Für den neuen ungerichteten Kreis gilt $C_{T',e'}=C_{T,e_1}$, zur Beibehaltung der Invariante muss \lstinline|class Circle| die gespeicherten Kanten rotieren. Für die Konstellation, dass $e_1$ im Gegensatz zu $e'$ nicht saturiert ist oder vice versa ändert sich zusätzlich die Richtung von $\bar{C}_{T',e'}$ im Vergleich zu $\bar{C}_{T,e_1}$. All dies leistet die Funktion \lstinline|Circle::rotateBy|:

\begin{lstlisting}
void Circle::rotateBy (size_t index, bool toReverse) {
  //rotate, such that the correct edge is on front
  std::rotate(edges.begin(),
              edges.begin()+index, edges.end());
  std::rotate(isResidual.begin(),
              isResidual.begin()+index, isResidual.end());
  //all but the first entry might have to be reversed
  if (toReverse) {
    //change direction of all edges
    for (size_t i = 0; i < this->length; i++) {
      std::swap(edges[i].first, edges[i].second);
      isResidual[i] = not isResidual[i];
    }
    std::reverse(edges.begin() + 1, edges.end());
    std::reverse(isResidual.begin() + 1, isResidual.end());
    }
}
\end{lstlisting}

Der einfache Fall a) aus \cref{fig:circles} ist damit abgehandelt, da $C_2$ unverändert bleibt. Dies gilt ebenfalls für Fall b), insofern $e'\notin E(C_{T,e_1})\cap E(C_{T,e_2})$. Andernfalls wird, wie \cref{fig:bypass} veranschaulicht, $C_{T',e_2}$ nun $e_1$ oder $\varphi(e_1)$ statt $e'$ enthalten.

\begin{figure}[!ht]\centering
    \includestandalone{tikz_circlesbypass}
    \caption{Ursprünglich sind $e_1$ und $e_2$ Nichtbaumkanten, die Orientierung von $e_1$ und $e'$ ist irrelevant. Der Kreis $\bar{C}_{T,e_2}$ geht von $x$ über $e_2$ nach $y$ (schwarze Kanten) zurück nach $x$ über $e'$ oder $\varphi(e')$ (rote Kanten). Im neuen Baum $T'$, der statt $e'$ die Kante $e_1$ enthält, führt $\bar{C}_{T',e_2}$ nun von $y$ nach $x$ über $e_1$ oder $\varphi(e_1)$ (blaue Kanten).}
    \label{fig:bypass}
\end{figure}

Sei $W:=E(C_{T,e_1})\cap E(C_{T,e_2})$ der Schnitt der beiden ungerichteten Kreise im alten Baum $T$. Dann gilt:
\begin{align}E(C_{T',e_2})&=\big(E(C_{T,e_2})\backslash W\big)\cup\big(E(C_{T,e_1})\backslash W\big)\nonumber\\
&=\big(E(C_{T,e_1})\cup E(C_{T,e_2})\big)\backslash W\label{eq:XOR}\end{align}

Gemäß \cref{eq:XOR} lässt sich also nach einer Iteration vom aufspannenden Baum $T$ zu $T'=T-\{e_1\}+\{e'\}$ die Kantenmenge jedes weiteren ungerichteten Kreises $C_{T,e_i}$ mit $i\neq1$  aktualisieren, indem wir eine XOR-Operation mit der alten Kantenmenge und $E(C_{T,e_1})=E(C_{T',e'})$ durchführen. Um $\bar{C}_{T',e_i}$ zu erhalten, muss der Kreis nur noch entsprechend $e_i$ gerichtet werden.

Wir werden nun die recht komplexe Funktion \lstinline|Circle::update|, die oben beschriebene Routine umsetzt, im Detail betrachten. Die Funktion bekommt \lstinline|Circle& c| übergeben, dies entspricht $\bar{C}_{T',e'}$. Zunächst suchen wir sowohl in unseren Kanten als auch in ihren umgedrehten Pendants nach $e'$. Dies ist nötig, da wir die Orientierung der Kreise zueinander nicht kennen.

Sind wir nicht fündig geworden, ist die Routine abgeschlossen, da es nichts zu tun gibt. Ansonsten erreichen wir den Zustand, dass die beiden gerichteten Kreise gegeneinander orientiert sind, also der Schnitt ihrer Kanten leer ist, indem wir gegebenenfalls \lstinline|c.rotateBy(0, true)| aufrufen. Damit ist gegeben, dass in \lstinline|Circle c| nach Notation von \cref{fig:bypass} der Weg von $y$ nach $x$ den blauen Kanten entspricht.

\begin{lstlisting}
void Circle::update(Circle& c) {
  bool iR = c.getIsResidual()[0];
  std::pair<size_t, size_t> edgeSame = c.getEdges()[0];
  std::pair<size_t, size_t> edgeReverse =
       std::make_pair(edgeSame.second, edgeSame.first);

  bool reversed = false;
  size_t index = 0;
  std::vector<std::pair<size_t, size_t>>::iterator
       it = edges.begin();
  //find edgeSame if existent
  for (index = 0; it != edges.end(); index++) {
    if (iR == isResidual[index] and
        edges[index] == edgeSame) {break;}
  it++;
  }
  //if edgeSame is in this->edges, reverse c,
  //do if-case and reverse back
  if (it != edges.end()) {
    c.rotateBy(0, true);
    reversed = true;
  }
  //else try to find edgeReverse
  else {
    it = edges.begin();
    for (index = 0; it != edges.end(); index++) {
      if (iR != isResidual[index] and 
          edges[index] == edgeReverse) {break;}
      it++;
    }
  }
\end{lstlisting}

Insofern wir $e'$ oder $\varphi(e')$ gefunden haben, zeigt der Iterator \lstinline|it| an dieser Stelle darauf. Indem wir von dieser Kante aus $x$ und $y$ suchen, finden wir heraus, welche Kanten in beiden zugrundeliegenden ungerichteten Kreisen enthalten sind. Insbesondere haben wir danach in \lstinline|size_t indexLeft| und \lstinline|indexRight| gespeichert, bis wohin bzw. ab wann wir die Kanten unseres bisherigen gerichteten Kreises beibehalten.

\begin{lstlisting}
  if (reversed or it != edges.end()) {
    std::vector<bool> toCopy (c.length, true);
    size_t indexLeft = index - 1, indexRight = index + 1;
    
    //take out all edges existent in both circles
    toCopy[0] = false; //first edge not even in tree
    //left from first edge
    for (; indexLeft > 0; indexLeft--) {
      //circles are in opposite directions,
      //so reverse the edge for existence check
      std::pair<size_t, size_t> checkEdge =
      std::make_pair(
           c.getEdges()[index - indexLeft].second,
           c.getEdges()[index - indexLeft].first);
      iR = c.getIsResidual()[index - indexLeft];
      if (iR != isResidual[indexLeft] 
          and edges[indexLeft] == checkEdge)
      {toCopy[index - indexLeft] = false;}
      else {break;}
    }
    //right from first edge
    for (; indexRight < this->length; indexRight++) {
      //circles are in opposite directions,
      //so reverse the edge for existence check
      std::pair<size_t, size_t> checkEdge =
      std::make_pair(
           c.getEdges()[c.length+index-indexRight].second,
           c.getEdges()[c.length+index-indexRight].first);
      iR = c.getIsResidual()[c.length+index-indexRight];
      if (iR != isResidual[indexRight]
          and edges[indexRight] == checkEdge)
      {toCopy[c.length + index - indexRight] = false;}
      else {break;}
    }
\end{lstlisting}

Anschließend konstruiere ich der Einfachheit halber die beiden Vektoren \lstinline|edges| und \lstinline|isResidual| komplett neu. Der neue Kreis besteht dann wie in \cref{fig:bypass} aus dem Weg von $e_2$ nach $y$, gefolgt von den neuen, im Bild blauen Kanten von $y$ nach $x$ und dem Schluss von $x$ zu $e_2$. Falls wir \lstinline|Circle c| umgedreht haben, machen wir dies als letztes noch rückgängig.
    
\begin{lstlisting}
    //construct new circle
    std::vector<std::pair<size_t, size_t>> edgesNew;
    std::vector<char> isResidualNew;
    
    //first part of old circle
    for (size_t i = 0; i <= indexLeft; i++) {
      edgesNew.push_back(this->edges[i]);
      isResidualNew.push_back(this->isResidual[i]);
    }
    //bypass
    for (size_t i = 0; i < c.length; i++) {
      if (toCopy[i]) {
        edgesNew.push_back(c.getEdges()[i]);
        isResidualNew.push_back(c.getIsResidual()[i]);
      }
    }
    //last part of old circle
    for (size_t i = indexRight; i < this->length; i++) {
      edgesNew.push_back(this->edges[i]);
      isResidualNew.push_back(this->isResidual[i]);
    }
    
    this->edges = edgesNew;
    this->isResidual = isResidualNew;
  }

  //reverse back
  if(reversed) {c.rotateBy(0, true);}
}
\end{lstlisting}

Mit diesen Datenstrukturen als Grundlage können wir nun den Algorithmus implementieren.

\section{Netzwerk-Simplex-Algorithmus}
Die Klasse \lstinline|class Algorithm| besteht nur aus einem Konstruktor und der Funktion \lstinline|Algorithm::solution|, die das Min-Cost-Flow-Problem entweder mit LC- oder HC-Initialisierung (siehe \cref{ch:init} für die Funktionsweise und \cref{ch:HCLC} für die Umsetzung) löst. Ein \lstinline|Algorithm a| wird mit einer konkreten Pivotfunktion erstellt, die aus einem \lstinline|std::vector<Circle>| einen Kreis auswählt und dessen Index zurückgibt.
\begin{lstlisting}
class Algorithm {
public:
  Algorithm (Network& _n, size_t (*_pivot)
                          (const std::vector<Circle>&))
    : n(_n), pivot(_pivot) {};

  //returns 0 if n cannot be solved
  //false for high cost edge initialization
  bool solution (bool modus = true);
}
\end{lstlisting}

Die Klasse speichert intern das Netzwerk, den Pivotalgorithmus, die Kreise, die \lstinline|id| des künstlichen Knoten sowie die Anzahl an Iterationen ab. Zusätzlich gibt es eine Funktion zur anfänglichen Erzeugung aller Kreise per Tiefensuche.

\begin{lstlisting}
class Algorithm {
private:
  intmax_t iterations = 0;
  Network& n;
  size_t (*pivot)(const std::vector<Circle>&);
  size_t artNode;
  std::vector<Circle> circles;

  //false if no optimization was possible
  bool optimize();
  //takes tree and creates circles;
  void createCircles(std::vector<Node> tree);
};
\end{lstlisting}

Der Kern der Klasse ist die Funktion \lstinline|Algorithm::optimize|, die eine Iteration durchführt oder \lstinline|false| zurückgibt, falls dies nicht mehr möglich und damit Phase 2 beendet ist. Nachdem Phase 1 abgeschlossen ist, wird dementsprechend die Schleife \lstinline|while (optimize());| ausgeführt.
 
Die Funktion aktualisiert zunächst die Werte \lstinline|flow| und \lstinline|costPerFlow| für die Kreise, insofern notwendig. Dann wählt sie durch den Pivotalgorithmus einen Kreis \lstinline|Circle c| für die nächste Iteration aus oder gibt \lstinline|false| zurück, falls es keinen negativen Kreis mehr gibt. \lstinline|c| wird um \lstinline|c.flow| augmentiert und gemäß \cref{ch:deg} wird die Kante $e'$ gewählt, die den Baum verlässt. Zum Schluss wenden wir \lstinline|Circle::update| für alle anderen Kreise an. Mit unseren bisher vorgestellten Datenstrukturen und Codeschnipseln ist alles bis auf die Wahl von $e'$ leicht implementierbar. Das nächste Kapitel beschäftigt sich mit der Umsetzung dieses Teilproblems.

\subsection{Stark zulässige Baumlösungen}
TODO: Fun Fact – um stark zulässige Baumlösungen zu erhalten, muss ich vermutlich doch den Baum speichern. Na ja.

\subsection{Pivotalgorithmen}
Die Pivotalgorithmen werden in meiner Implementierung in \lstinline|PivotAlgorithms.h| gesammelt. Ein Pivotalgorithmus gibt einen ungültigen Index zurück, falls es keinen negativen Kreis gibt.

\begin{lstlisting}
//return index of chosen circle or circles.size() if
//no circle can be chosen

//returns a random circle with negative cost
size_t pivotRandom(const std::vector<Circle>& circles);
//returns most negative circle
size_t pivotMaxVal(const std::vector<Circle>& circles);
//returns max |flow*costperflow|
size_t pivotMaxRev(const std::vector<Circle>& circles);
\end{lstlisting}

Beispielhaft schauen wir uns noch die Umsetzung von \lstinline|pivotMaxVal| an:

\begin{lstlisting}
size_t pivotMaxVal(const std::vector<Circle>& circles) {
  intmax_t mini = 0;
  size_t index = circles.size();
  for (size_t i = 0; i < circles.size(); i++) {
    intmax_t value = circles[i].costPerFlow;
    if (value < mini) {mini = value; index = i;}
  }
  return index;
}
\end{lstlisting}

\subsection{Initialisierung}\label{ch:HCLC}
Wir haben nun alle essentiellen Bestandteil der Implementierung von Phase 2 des Netzwerk-Simplex-Algorithmus gesehen. Zum vollständigen Algorithmus fehlt uns nur noch Phase 1, wie in \cref{ch:init} beschrieben. Zu Beginn legen wir je nach Modus die Kosten der künstlichen Kanten fest.

\begin{lstlisting}[escapechar=ß]
bool Algorithm::solution (bool modus) {
  iterations = 0;
  //high cost edge
  intmax_t maxCost = 0;
  if (modus == false) {
    for (autoß\footnotemarkß edge : n.getEdges()) {
      if (edge.second.cost > maxCost) {
        maxCost = edge.second.cost;
      }
    }
  }
  //toggle all edges
  else {n.toggleCost();}
  maxCost = maxCost*n.getNoOfNodes() + 1;
\end{lstlisting}

\footnotetext{\lstinline|const std::pair<const std::tuple<size_t, size_t, bool>, Edge>&|}

Nun legen wir den künstlichen Knoten und die künstlichen Kanten an, hier beispielhaft für die Quellen:
\begin{lstlisting}
  artNode = n.addNode(0);
  Node artNode_Tree = Node(artNode, 0);

  for (size_t source : n.sources) {
    n.addEdge(Edge(source, artNode, maxCost,
                   n.sumSource + 1));
  }
\end{lstlisting}

Die Funktion \lstinline|bool Algorithm::solution| gibt sofort \lstinline|false| zurück, falls der übergebene Graph $G$ die Gleichung $\sum_{v\in V(G)} b(v)=0$ nicht erfüllt. Somit können wir bei der Erzeugung des initialen Flusses $f_0$ Wege von Quellen zu Senken über den künstlichen Knoten augmentieren, bis der b-Wert eines der beiden Endknoten erfüllt ist, und dann mit der nächsten Quelle bzw. Senke iterieren.

\begin{lstlisting}
  //get that flow started
  std::vector<size_t>::iterator iSo = sources.begin(),
                                iSi = sinks.begin();
  while (iSo != sources.end()) {
    if (std::abs(n.getNodes().find(*iSo)->second.b_value)
     <= std::abs(n.getNodes().find(*iSi)->second.b_value)){
      std::vector<size_t> path = {*iSo, artNode, *iSi};
      n.addFlow(path,
        std::abs(n.getNodes().find(*iSo)->second.b_value));
      iSo++;
    }
    else {
      std::vector<size_t> path = {*iSo, artNode, *iSi};
      n.addFlow(path,
        std::abs(n.getNodes().find(*iSi)->second.b_value));
      iSi++;
    }
  }
\end{lstlisting}

Die Initialisierung ist an dieser Stelle abgeschlossen, nun können wir die Instanz lösen.

\begin{lstlisting}
  //complete tree and create circles
  createCircles(tree);

  //solve with artificial node
  while (optimize());
\end{lstlisting}

Ist \lstinline|modus = false|, dann befinden wir uns in HC und sind an dieser Stelle fertig. Kann der künstliche Knoten gelöscht werden, ist die Instanz lösbar und die Lösung im Netzwerk abgespeichert, andernfalls gibt \lstinline|solution| den Wert \lstinline|false| zurück.

Für LC wissen wir zunächst nur, ob die Instanz lösbar ist oder nicht. Wenn sie lösbar ist, können wir den künstlichen Knoten löschen, über \lstinline|Network::toggleCost| die originalen Kosten wiederherstellen und mit \lstinline|optimize| eine optimale Lösung finden.

Um die dafür notwendige stark zulässige Baumlösung auf dem Netzwerk ohne künstlichen Knoten zu finden, gibt es zwei Ansätze: Ein Weg ist es, einfach auf dem Graphen einen neuen Baum $T'$ zu bestimmen, der mit dem aktuellen Fluss $f$ eine zulässige Baumlösung $(T',f)$ bildet. In meiner Umsetzung führen wir dagegen vor der Löschung des künstlichen Knoten degenerierte Iterationen durch, bis dieser ein Blatt ist. Dies schien mir der kohärente Ansatz zu sein.

Dafür gehen wir über den Vektor \lstinline|circles| und suchen nach Kreisen, in denen zwei Baumkanten mit dem künstlichen Knoten benachbart sind. Sei $e$ diejenige der beiden Kanten, die den Baum verlässt. $e$ kann nur in schon betrachteten Kreisen enthalten sein, wenn deren identifizierende Kante mit dem künstlichen Knoten benachbart ist; diese Kreise werden jedoch später gelöscht. Also genügt es, nur Kreise über \lstinline|update| zu aktualisieren, die noch nicht betrachtet wurden, womit folgende Subroutine terminiert.

\begin{lstlisting}
  for (std::vector<Circle>::iterator it = circles.begin();
       it != circles.end(); it++) {
  //find edge over artificial node if existent
  size_t i = 1;
  for (; i < it->size(); i++) {
    if (it->getEdges()[i].first == artificialNode or
        it->getEdges()[i].second == artificialNode)
      {break;}
  }
  if (i < it->size()) {
    //since n is feasible, one of the both cases occures

    //not residual <==> flow == 0
    //new first edge, same direction
    if (not it->getIsResidual()[i])
      {it->rotateBy(i, false);}
    //new first edge, but reversed direction
    else {it->rotateBy(i, true);}

    //circles before this one don't need an update 
    for (std::vector<Circle>::iterator otherCircle = it+1;
         otherCircle != circles.end(); otherCircle++)
      {otherCircle->update(*it);}
    iterations++;
  }
\end{lstlisting}

Zuletzt werden der künstliche Knoten und wie angedeutet die $|V(G)|-1$ Kreise gelöscht, die am künstlichen Knoten beginnen, bevor die finale Lösung berechnet wird. Ich benutze an dieser Stelle eine Lambdafunktion, um die entsprechenden Kreise herauszufiltern.

\begin{lstlisting}
  //now remove all circles beginning at the artificial node
  circles.erase(std::remove_if(circles.begin(), 
                               circles.end(),
    [this](const Circle& c)
    {return c.getEdges()[0].first == artificialNode or
            c.getEdges()[0].second == artificialNode;}
    ), circles.end());

  while (optimize());
  return true;
}
\end{lstlisting} 