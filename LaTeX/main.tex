\documentclass[a4paper,twoside]{report}
\author{Max~Kanold}
\title{Experimentelle Ergebnisse zum Network-Simplex-Algorithmus}

\usepackage[utf8]{inputenc}
\usepackage[T1]{fontenc}

\usepackage{mathtools}

\usepackage[hidelinks]{hyperref}
% or colorlinks

\usepackage{amssymb,amsthm}

\theoremstyle{plain}
\newtheorem{thm}{Theorem}

\theoremstyle{definition}
\newtheorem{defn}[thm]{Definition}

\usepackage[ngerman]{babel}
\usepackage{csquotes}
\usepackage{anyfontsize}
\usepackage[babel]{microtype}



\begin{document}
\maketitle
\tableofcontents

\newpage
\chapter{Einführung}
Bla. Zum Beispiel in Kapitel \ref{prog} habe ich programmiert.

\newpage
\chapter{Network-Simplex-Algorithmus}
Das Simplex-Verfahren, zu welchem eine Einführung in \cite{NSAbook} gefunden werden kann, löst Lineare Programme in der Praxis sehr schnell, obwohl die Worst-Case-Laufzeit nicht polynomiell ist. Jedes Netzwerkproblem lässt sich als Lineares Programm darstellen und somit durch das Simplex-Verfahren lösen, durch die konkrete Struktur solcher Probleme genügt jedoch der vereinfachte Network-Simplex-Algorithmus. Auch für diesen gibt es exponentielle Instanzen (siehe \cite{Exponential}), in der Praxis wird er trotzdem vielfach verwendet.

\section{Min-Cost-Flow-Problem}
\begin{defn}Ein \textbf{Netzwerk} $(G,u,c,b)$ und so weiter.\end{defn}
Sind die Kapazitäten unbeschränkt, ist das Problem als \emph{Transportproblem} bekannt.

Für diese Bachelorarbeit wurde angenommen, dass $u$ und $c$ auf $\mathbb{N}$ sowie $b$ auf $\mathbb{Z}$ abbildet, um Gleitkommazahlungenauigkeit zu vermeiden. Durch eine entsprechende Skalierung des Problems können die Funktionen nach $\mathbb{R}$ hinreichend genug angenähert werden. Zusätzlich wird davon ausgegangen, dass $\sum_{v\in V(G)} b(v) = 0$ ist, Angebot und Nachfrage also ausgeglichen sind. Des Weiteren sind in der konkreten Implementierung keine parallelen Kanten vorgesehen.

\begin{defn}Eine Lösung TODO \textbf{Min-Cost-Flow-Problem} sucht für ein gegebenes Netzwerk $(G,u,c,b)$ eine \end{defn}


\section{Algorithmus}

\section{Umsetzung} \label{prog}
Hier beginnt mein schönes Werk \ldots

\subsection{Spezielle Konstrukte}
\ldots und hier endet es.

\subsubsection{Die Klasse Circle}
Kreise halt.\cite{NSAbook}

\subsubsection{Der Rest halt}
Kleinkram.

\newpage
\chapter{Experimentelle Ergebnisse}
Alle scheiße.

\newpage
\chapter{Ausblick}
La la la.

\bibliography{bibliography}{}
\bibliographystyle{ieeetr}

\end{document}