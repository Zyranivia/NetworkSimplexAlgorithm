\documentclass[11pt,a4paper,twoside,ngerman,openany]{scrbook}
\KOMAoptions{BCOR=6mm}
\author{Max~Kanold}
\title{Eine graphentheoretische Herleitung und Implementierung des Netzwerk-Simplex-Algorithmus}

%\pagestyle{plain}
\usepackage{scrhack}

\usepackage[utf8]{inputenc}
\usepackage[T1]{fontenc}

\usepackage[ngerman]{babel}
\usepackage{csquotes}
\usepackage{anyfontsize}
\usepackage[babel]{microtype}

%clever space after own command
\usepackage{xspace}
%\-/ can be used to allow LaTeX hyphenation in dashed words
\usepackage[shortcuts]{extdash}

\usepackage[hidelinks]{hyperref}
% or colorlinks
\usepackage{mathtools}
\usepackage{mathrsfs}
\usepackage{amsmath}
\usepackage{amsthm}
\usepackage{amssymb}
% restate a theorem
\usepackage{thmtools}
\usepackage{thm-restate}
% in between the refs to suppress warning
\DeclareMathOperator*{\argmin}{arg\,min}
\DeclareMathOperator*{\argmax}{arg\,max}

\usepackage[nameinlink]{cleveref}

\usepackage[shortlabels]{enumitem}
\setdescription{font=\normalfont}

\usepackage{standalone}
\usepackage{pgf, tikz}
\usetikzlibrary{arrows, automata}

\theoremstyle{plain}
\newtheorem{thm}{Theorem}
\newtheorem{lem}[thm]{Lemma}
\newtheorem{kor}[thm]{Korollar}

\theoremstyle{definition}
\newtheorem{defn}[thm]{Definition}

\newtheorem*{anm}{Anmerkung}
\newtheorem*{nota}{Notation}

\usepackage{listings}
\lstset{language=C++,
    basicstyle=\ttfamily,
    keywordstyle=\color{blue}\ttfamily,
    commentstyle=\color{gray}\ttfamily,
    morekeywords={size_t, intmax_t, uintmax_t},
    deletekeywords={delete}
}

\crefname{thm}{Theorem}{Theoreme}
\crefname{lem}{Lemma}{Lemmata}
\crefname{defn}{Definition}{Definitionen}
\crefname{kor}{Korollar}{Korollare}
\crefname{enumi}{Eigenschaft}{Eigenschaften}

\newcommand{\obda}{o.\,B.\,d.\,A. }
\newcommand{\Obda}{O.\,B.\,d.\,A. }
\newcommand{\cpp}{C\texttt{++}\xspace}
\newcommand{\cppelf}{C\texttt{++}11\xspace}

%TODO necessary?
\newenvironment{tightcenter}{%
    \setlength\topsep{0pt}
    \setlength\parskip{0pt}
    \begin{center}
    }{%
    \end{center}
}

\begin{document}
\frontmatter
\maketitle
\cleardoublepage
\tableofcontents
\cleardoublepage

\mainmatter
\chapter{Einführung}
Das Simplex-Verfahren, zu welchem eine Einführung in \cite{NSAbook} gefunden werden kann, löst Lineare Programme in der Praxis sehr schnell, obwohl die Worst-Case-Laufzeit nicht polynomiell ist. Jedes Netzwerkproblem lässt sich als Lineares Programm darstellen und somit durch das Simplex-Verfahren lösen, durch die konkrete Struktur solcher Probleme genügt jedoch der von \cite[Dantzig, 1951]{erf1} und \cite[Orden, 1956]{erf2} vereinfachte Netzwerk-Simplex-Algorithmus. Auch für diesen gibt es exponentielle Instanzen (siehe \cref{ch:lit}), in der Praxis wird er trotzdem vielfach verwendet.

TODO: Die Einführung wird ohnehin ausgebaut. Irgendwie erwähne ich auch unauffällig, dass in dieser Bachelorarbeit die Laufzeit des Algorithmus als Iterationsanzahl definiert ist, obwohl in der Praxis natürlich die Gesamtgeschwindigkeit zählt.

Sämtliche Bilder von Graphen im Verlauf dieser Bachelorarbeit sind nach folgender Legende zu lesen. Teilweise werden nur einzelne Werte angezeigt, Ausnahme ist die Kapazität, die nie ohne Flusswerte auftritt.
\begin{itemize}\itemsep0em
    \item b-Wert: schwarz im Knoten
    \item Fluss: \textcolor{blue}{blau} auf Kante
    \item Kapazität: schwarz auf Kante, mit Schrägstrich (/) vom Fluss abgetrennt
    \item Kosten: \textcolor{red}{rot} auf Kante, falls notwendig mit Komma (,) abgetrennt
\end{itemize}

\chapter{Netzwerk-Simplex-Algorithmus}\label{ch:NSA}
Zunächst führen wir in \cref{ch:MCF} das Transportproblem und dessen Verallgemeinerung auf beschränkte Kapazitäten ein. Gleichzeitig geben wir die Einschränkungen an, auf denen insbesondere der Programmierteil dieser Bachelorarbeit fußt. Die Beschreibung des Netzwerk-Simplex-Algorithmus in \cref{ch:alg} orientiert sich zuerst an \cite[S. 291\,ff.]{NSAbook} zur Lösung des Transportproblems, danach wird in \cref{ch:alg2} der Algorithmus anhand von \cite[S. 353\,ff.]{NSAbook} auf den allgemeinen Fall erweitert.

Wir werden nur endliche Graphen betrachten. Außerdem werden alle Graphen einfach sein, das heißt, sie weisen weder mehrfache Kanten noch Schleifen auf. Der später eingeführte Residualgraph wird weiterhin keine Schleifen besitzen, doppelte Kanten können unter Umständen vorkommen.

\section{Min-Cost-Flow-Problem}\label{ch:MCF}
\begin{defn}Ein \textbf{Netzwerk} ist ein Tupel $(G,b,c,u)$ aus einem gerichteten Graph $G = (V,E)$ und den Abbildungen $b \colon V\rightarrow\mathbb{R}$, $c \colon E\rightarrow\mathbb{R}$ sowie $u \colon E\rightarrow\mathbb{R}_{\geq 0}\cup \infty$. Wir bezeichnen $b$ als \emph{b-Wert-Funktion}, $c$ als \emph{Kostenfunktion} und $u$ als \emph{Kapazitätsfunktion}.\end{defn}
\begin{anm}Knoten mit positivem b-Wert werden als \emph{Quellen}, solche mit negativem als \emph{Senken} bezeichnet. Knoten mit neutralem b-Wert werden \emph{Transitknoten} genannt. Ein ungerichteter Graph kann durch das Verwandeln jeder Kante $\{v,w\}$ in zwei Kanten $(v,w)$ und $(w,v)$ zu einem gerichteten modifiziert werden.\end{anm}

\begin{defn}\label{DefMaxFlow}Ein \textbf{maximaler Fluss} auf einem Netzwerk $(G=(V,E),b,c,u)$ ist eine Abbildung $f \colon E\rightarrow\mathbb{R}_{\geq 0}$, die folgende Eigenschaften erfüllt:
\begin{align}
&\text{(i)}&&\forall e\in E \colon f(e)\leq u(e)\nonumber\\
&\text{(ii)}&&\forall v\in V \colon \sum_{\mathclap{(w,v)\in E}} f((w,v)) - \sum_{\mathclap{(v,w)\in E}} f((v,w)) + b(v) = 0\label{DefMaxFlowII}
\end{align}
Die \emph{Kosten} von $f$ betragen $c(f) = \sum\limits_{e\in E} f(e)\cdot c(e)$.
\end{defn}

\begin{defn}Sei $N=(G,b,c,u)$ ein Netzwerk. Als \textbf{Min-Cost-Flow-Problem} bezeichnen wir die Suche nach einem maximalen Fluss $f$ auf $N$ mit minimalen Kosten. Die vereinfachte Version, bei der die Kapazitätsfunktion unbeschränkt ist, also $u(e)=\infty$ für alle $e\in E(G)$ gilt, nennen wir \textbf{Transportproblem}.\end{defn}

In dieser Bachelorarbeit wird angenommen, dass $b$ auf $\mathbb{Z}$ sowie $c$ und $u$ auf $\mathbb{N}_{\geq0}$ abbilden, um Gleitkommazahlungenauigkeit beim Programmieren zu vermeiden. Durch eine entsprechende Skalierung des Problems können Funktionen nach $\mathbb{R}$ bzw. $\mathbb{R}_{\geq 0}$ hinreichend genug angenähert werden. Es wird davon ausgegangen, dass die b-Werte der Quellen und Senken ausgeglichen sind, also $\sum_{v\in V(G)} b(v) = 0$ gilt. Sollte die Summe der Senken überwiegen, ist die Instanz unlösbar. Der Fall einer zu großen Summe der Quellen kann durch eine Dummy-Senke\footnotemark{} abgefangen werden.

\footnotetext{Gemäß \cite[S. 454]{LP&NF2010} fügen wir $G$ eine zusätzliche Senke $s$ hinzu, die mit allen Quellen $q_i$ über eine Kante $e_i=(q_i,s)$ verbunden ist und einen b-Wert von $b(s)=-\sum_{v\in V(G)} b(v)$ zugewiesen bekommt. Für die Kanten gilt $c(e_i)=0$ und $u(e_i)=\infty$.}

Durch die auf nicht-negative Werte eingeschränkte Kostenfunktion hat kein maximaler Fluss negative Kosten. Damit gibt es keine unbeschränkten Instanzen. Unbeschränkte Kapazitäten können somit in der konkreten Implementierung durch $1+0,5\cdot\sum_{v\in V} |b(v)|$ abgeschätzt werden, ohne dass die Lösungsmenge verändert wird. Alle Netzwerke werden als zusammenhängend angenommen, da die Zusammenhangskomponenten einer Instanz einzeln gelöst werden können. Der programmierte Algorithmus verlangt keinen Zusammenhang.

\section{Der grundlegende Algorithmus}\label{ch:alg}
Um die Funktionsweise des Netzwerk-Simplex-Algorithmus zu definieren, benötigen wir zuerst einige grundlegende Definitionen der Graphentheorie. Im Nachfolgenden sind die wichtigsten in Kürze aufgeführt, für eine vollständige Einführung sei der geneigte Leser auf \cite{Alma} verwiesen.

\begin{defn}Der einem gerichteten Graphen $G=(V,E)$ \textbf{zugrundeliegende ungerichtete Graph} $G'=(V,E')$ ist definiert durch:
\begin{equation*}\{v,w\}\in E' \iff (v,w) \in E \lor (w,v) \in E\end{equation*} \end{defn}
\begin{anm}Nach dieser Definition bleibt der zugrundeliegende Graph einfach. Eine gängige Alternative behält alle Kanten und entfernt nur die Richtung, dies ist für unsere Zwecke jedoch nicht praktikabel.\end{anm}

\begin{defn}Ein \textbf{Baum} $T$ ist ein ungerichteter, zusammenhängender und kreisfreier Graph. Ein \textbf{Wald} ist ein Graph, bei dem jede Zusammenhangskomponente ein Baum ist.\\
Ein Teilgraph $T=(V',E')$ eines ungerichteten Graphen $G=(V,E)$ heißt \textbf{aufspannender Baum}, wenn T ein Baum und $V'=V$ ist.\end{defn}

\begin{anm}Sprechen wir bei einem gerichteten Graphen $G$ über einen Wald bzw. (aufspannenden) Baum, so bezieht sich das stets auf einen Teilgraphen $T$ von $G$, dessen zugrundeliegender ungerichteter Graph ein Wald bzw. (aufspannender) Baum des $G$ zugrundeliegenden ungerichteten Graphen ist.\end{anm}

\begin{defn}Sei $N=(G,b,c,u)$ eine Instanz des Transportproblems. Ein aufspannender Baum $T$ von $G$ und ein maximaler Fluss $f$ auf $N$ bilden eine \textbf{zulässige Baumlösung} $(T,f)$, wenn für alle Kanten $e\in E(G)\backslash E(T)$ außerhalb des Baumes $f(e) = 0$ gilt.\end{defn}

\begin{nota}Sei $G=(V,E)$ ein Graph und $V'\subseteq V$ eine Teilmenge der Knoten. Der \emph{induzierte Teilgraph} $G[V']=(V',\{(v,w)\in E\mid v\in V'\land w\in V'\})$ enthält alle Knoten aus $V'$ und alle Kanten zwischen ihnen, die schon in $E$ vorhanden waren.\end{nota}

%bug with restatable, see https://tex.stackexchange.com/questions/111639/extra-spacing-around-restatable-theorems
\vspace{-1.9ex}
\begin{restatable}{lem}{TreeUnique}\label{TreeUnique}Jede zulässige Baumlösung $(T,f)$ ist eindeutig durch den aufspannenden Baum $T$ definiert.\end{restatable}
\begin{proof}Sei $(T,f)$ eine zulässige Baumlösung einer Instanz des Transportproblems. Wir führen eine Induktion über die Anzahl der Knoten des Baumes durch. Die zulässige Baumlösung zum leeren Baum ist offensichtlich.

Für jedes Blatt von $l\in V(T)$ sei $e_l\in E(T)$ die Kante in $T$ zwischen dem Knoten $l$ und seinem eindeutigen Nachbarknoten $k_l$. Für alle Blätter $l$ ist der Wert von $f(e_l)$ nach \cref{DefMaxFlowII} eindeutig. Wir betrachten nun die eingeschränkte Knotenmenge $V'=\{v\in V(T)\mid v\text{ ist kein Blatt in }T\}$. Da jeder Baum mindestens zwei Blätter besitzt, ist $|V'|<|V(T)|$. Sei $T'=T[V']\subsetneq T$ der durch $V'$ induzierte Teilbaum von $T$. Damit sind $E(T')$ genau die Kanten, für die $f$ noch nicht bestimmt wurde.

Sei $N'=(G'=G[V'],b',c_{|E(G')},u_{|E(G')})$ das auf $V'$ eingeschränkte Netzwerk $N$ mit der folgendermaßen angepassten b-Wert-Funktion $b'$:
\begin{equation}\label{eq:newB}
\forall v'\in V'\colon\quad b'(v')=b(v')+\sum_{\substack{l\in V(T)\backslash V':\\(l,v')\in E(T)}} b(l)-\sum_{\substack{l\in V(T)\backslash V':\\(v',l)\in E(T)}} b(l)\end{equation}

Die b-Werte von Blattnachbarn bezogen auf Blätter in $T$ werden angepasst, alle anderen bleiben identisch. Nun bekommen wir durch die Induktionsbehauptung für den aufspannenden Baum $T'$ von $G'$ einen eindeutigen Fluss $f'$. Wir setzen für alle Kanten $e\in E(T')$ den Fluss $f(e):=f'(e)$. Nach \cref{eq:newB} ist $f$ ein maximaler Fluss von $N$, außerdem ist $f$ nach Konstruktion eindeutig.\end{proof}

Wie \cref{fig:BL} veranschaulicht, gibt es maximale Flüsse, zu denen wir keine zulässige Baumlösung finden können. Auch wenn wir uns auf die maximalen Flüsse beschränken, zu denen es zulässige Baumlösungen gibt, gilt die Gegenrichtung, also dass $T$ eindeutig aus $f$ bestimmt werden kann, gemäß \cref{fig:BL} nicht.

\begin{figure}[!ht]\centering
\includestandalone{tikz_BL}
\caption{Links ein maximaler Fluss ohne zulässige Baumlösung, rechts ein maximaler Fluss mit uneindeutiger zulässiger Baumlösung.}
\label{fig:BL}
\end{figure}	

Wie wir in \cref{TP} zeigen werden, existiert für das Min-Cost-Flow-Problem immer eine Lösung, die auch eine zulässige Baumlösung ist. Die dem Algorithmus zugrundeliegende Idee ist es, über die Bäume zulässiger Baumlösungen mit sinkenden Kosten zu iterieren. Der Übergang basiert dabei auf dem Augmentieren negativer Kreise, eine Methode, für die das Konzept des Residualgraphen hilfreich ist.

\begin{defn}\label{defRes}Sei $N=(G=(V,E),b,c,u)$ ein Netzwerk mit einem maximalen Fluss $f$. Die \textbf{Residualkante} $\bar{e}$ einer Kante $e=(v,w)\in E$ verläuft von $w$ nach $v$. Sei $\bar{E}=\{\bar{e}\mid e\in E\}$ die Menge aller Residualkanten. Die \textit{Residualkapazität} $u_f\colon\bar{E}\rightarrow\mathbb{N}_{\geq0}$ ist bestimmt durch $u_f(\bar{e})=f(e)$, die \textit{Residualkosten} $c_f\colon\bar{E}\rightarrow\mathbb{Z}$ durch $c_f(\bar{e})=-c(e)$.

Der \textbf{Residualgraph} $R$ ist ein Tupel $R_{N,f}=(\bar{G}=(V,E\amalg\bar{E}),\bar{f},b,\bar{c},\bar{u})$ mit dem gerichteten Multigraphen $\bar{G}$ aus der bisherigen Knotenmenge und der disjunkten Vereinigung von Kanten und Residualkanten, dem maximalen Fluss $\bar{f}$ mit $\bar{f}(e)=f(e)$ für alle $e\in E$ und $\bar{f}(\bar{e})=0$ für alle $\bar{e}\in\bar{E}$, der b-Wert-Funktion $b$ wie gehabt, der Kostenfunktion $\bar{c}=c\amalg c_f\colon E\amalg\bar{E}\rightarrow\mathbb{Z}$ und der Kapazitätsfunktion $\bar{u}=u\amalg u_f\colon E\amalg\bar{E}\rightarrow\mathbb{N}_{\geq0}\cup\infty$.\end{defn}

\begin{nota}Mit $\varphi\colon E\leftrightarrow\bar{E}$ bezeichnen wir die kanonische Bijektion zwischen den Kanten und ihren Residualkanten.\end{nota}

Wenn wir uns in einem Residualgraphen $R_{N,f}$ befinden und dort den Fluss einer Residualkante $\bar{e}$ um $0\leq\delta\leq u_f(\bar{e})$ erhöhen, so wird in Wirklichkeit der Fluss $f(\varphi(\bar{e}))$ um $\delta$ reduziert. Nach obiger Definition ist sichergestellt, dass $f(\varphi(\bar{e}))-\delta\geq0$.

Verändern wir den maximalen Fluss $f$ auf nur einer Kante, so ist die resultierende Funktion $f'$ kein maximaler Fluss mehr. Deswegen werden wir stattdessen gerichtete Kreise $C$ im Residualgraph $R_{N,f}$ zu einer Abbildung $f'$ augmentieren, das heißt, wir erhöhen den Fluss auf allen Kanten $e\in E(C)$ um einen festen Betrag $\delta\in\mathbb{N}_{\geq0}$, ohne dabei Kapazitätsschranken zu verletzen. Damit ist \cref{DefMaxFlowII} für maximale Flüsse weiterhin erfüllt:
\begin{alignat*}{4}
\forall v\in V(C)\colon&&&\sum_{\mathclap{(w,v)\in E(G)}} f'((w,v)) &&-\sum_{\mathclap{(v,w)\in E(G)}} f'((v,w)) &&+ b(v)\\
&=\quad&&\sum_{\mathclap{(w,v)\in E(G)}} f((w,v)) + \delta &&-\sum_{\mathclap{(v,w)\in E(G)}} f((v,w)) - \delta &&+ b(v) = 0\end{alignat*}
Also ist $f'$ ein maximaler Fluss. Seine Kosten betragen $c(f')=c(f) + \delta\cdot c(C)$.

\begin{lem}\label{negKreis}Sei $N$ eine Instanz des Transportproblems, $R_{N,f}$ ihr Residualgraph und $C$ ein gerichteter Kreis in $R_{N,f}$ mit negativen Kosten. Der größte Wert $\delta$, um den $C$ augmentiert werden kann, ist endlich, und nach der Augmentierung um $\delta$ zum neuen maximalen Fluss $f'$ existiert eine Residualkante $\bar{e}\in E(C)$, sodass die korrespondierende Kante einen Fluss von $f'(\varphi(e))=0$ besitzt. 
\end{lem}
\begin{proof}Sei $N=(G,b,c,u)$ ein Instanz des Transportproblems mit maximalen Fluss $f$ und $C$ ein negativer Kreis in $R_{N,f}$. Wir setzen $\delta:=\min_{e\in E(C)}\{\bar{u}(e)-\bar{f}(e)\}$. Da $c\colon E(G)\rightarrow\mathbb{N}_{\geq0}$ in die natürlichen Zahlen abbildet, enthält jeder negative Kreis in $R_{N,f}$ mindestens eine Residualkante $\bar{e}$. Damit ist $0\leq\delta\leq u_f(\bar{e})<\infty$.

Wir haben $\delta$ unter allen zulässigen Werten größtmöglich gewählt. Da alle Kanten $e\in E(G)$ unbeschränkte Kapazität haben und ihr Fluss $f(e)$ endlich ist, gibt es eine Residualkante $\bar{e}$ mit $u_f(\bar{e})=\delta$. Nachdem wir $C$ um $\delta$ zu einem maximalen Fluss $f'$ augmentiert haben, ist der neue Fluss auf der korrespondierenden Kante $e:=\varphi(\bar{e})$
\begin{equation*}
f'(e)=f(e)-\delta=f(e)-u_f(\bar{e})=f(e)-f(e)=0.\qedhere
\end{equation*}\end{proof}

\begin{nota}Sei $f$ ein maximaler Fluss für ein Netzwerk $(G=(V,E),b,c,u)$ und $H=(V'\subseteq V, E'\subseteq E)$ ein Teilgraph von $G$. Mit $H_f=(V',\{e\in E' \mid f(e) \neq 0\})$ bezeichnen wir den Graph aller durchflossenen Kanten von $H$.\end{nota}

Dank \cref{negKreis} wissen wir, dass nach maximaler Augmentierung eines negativen Kreises $C$ alle seine durchflossenen Kanten $C_f$ einen Wald bilden. Damit können wir nun zeigen, dass eine zulässige Baumlösung $(T,f)$ mit einem maximalen Fluss $f$ minimaler Kosten existiert.

\begin{thm}\label{BLex}Sei $N$ eine Instanz des Transportproblems mit einem maximalen Fluss $f$. Es existiert ein maximaler Fluss $\hat{f}$, sodass $c(\hat{f})\leq c(f)$ ist und eine zulässige Baumlösung $(\hat{T},\hat{f})$ existiert.\end{thm}
\begin{proof}Sei $N=(G,b,c,u)$ ein Instanz des Transportproblems mit maximalen Fluss $f$. Wir werden $f$ zu einem maximalen Fluss $\hat{f}$ umwandeln, sodass $G_{\hat{f}}$ ein Wald ist. Für $\hat{f}$ finden wir dann leicht eine zulässige Baumlösung. Wenn wir für die endlich vielen maximalen Zwischenflüsse $f=f_0,f_1,f_2,\ldots,f_n=\hat{f}$ sicherstellen, dass $c(f_{i+1})\leq c(f_i)$ ist, gilt auch $c(\hat{f})\leq c(f)$.

Betrachte einen maximalen Fluss $f_i$. Wenn $G_{f_i}$ ein Wald ist, setzen wir $\hat{f}:=f_i$ und sind fertig. Ansonsten gibt es einen ungerichteten Kreis $C\subseteq G_{f_i}$. Betrachte die beiden dazugehörigen, gerichteten, kantendisjunkten Kreise $C_1$ und $C_2$ in $R_{N,f_i}$. Nach \cref{defRes} gilt $c(C_1)=-c(C_2)$. Wir werden einen der Kreise so augmentieren, dass $|E(G_{f+1})|<|E(G_f)|$ ist.
\begin{itemize}[leftmargin=!,labelwidth=\widthof{Fall 1:}]
\item[Fall 1:] $c(C_1)=0$\\
	Mindestens einer der beiden Kreise enthält eine Residualkante, sei dies \obda $C_1$. Augmentiere nun $C_1$ analog zum Beweis von \cref{negKreis} größtmöglich zu einem maximalen Fluss $f_{i+1}$. Damit ist $C\nsubseteq G_{f_{i+1}}$ und $c(f_{i+1}) = c(f_i)$.
\item[Fall 2:] $c(C_1)\neq0$\\
	\Obda sei $c(C_1)<0$. Nach \cref{negKreis} erhalten wir einen maximalen Fluss $f_{i+1}$, sodass $C\nsubseteq G_{f_{i+1}}$ und $c(f_{i+1}) = c(f_i) + \delta\cdot c(C_1)<c(f_i)$.
\end{itemize}

Damit ist $G_{f_{i+1}}\subsetneq G_{f_i}$ sowie $c(f_{i+1})\leq c(f_i)$. Nach endlich vielen Iterationen erhalten wir somit $\hat{f}$.
\end{proof}

\begin{kor}\label{TP}Für jede lösbare Instanz des Transportproblems existiert eine zulässige Baumlösung $(T,f)$, sodass der maximale Fluss $f$ minimale Kosten hat.\qed\end{kor}

\begin{nota}Sei $N=(G,b,c,u)$ ein Netzwerk mit maximalen Fluss $f$, $T$ ein aufspannender Baum von $G$, $R_{N,f}=(\bar{G},\bar{f},b,\bar{c},\bar{u})$ der Residualgraph und $e\in E(\bar{G})\backslash E(T)$ eine weitere Kante. Mit $C_{T,e}$ bezeichnen wir den eindeutigen Teilgraph von $T\cup\{e\}$, dessen zugrundeliegender Graph ein Kreis ist, und mit  $\bar{C}_{T,e}$ den eindeutigen gerichteten Kreis in $\bar{G}$ zu $C_{T,e}$, der $e$ enthält.\end{nota}

Sei $N=(G,b,c,u)$ eine Instanz des Transportproblems. Wie beim Simplex-Algorithmus gibt es beim Netzwerk-Simplex-Algorithmus zwei Phasen. In der ersten wird eine initiale zulässige Baumlösung $(T_0,f_0)$ auf $N$ erzeugt; die beiden etablierten Vorgehensweisen werden in \cref{ch:init} beschrieben. Die Problematik einer Instanz ohne Lösung wird ebenfalls dort behandelt.

Die zweite Phase iteriert folgende Vorgehensweise: Betrachte Iteration $i$ mit zulässiger Baumlösung $(T_i,f_i)$. Wähle einen negativen Kreis $\bar{C}_{T,e}$ mit $e\in E(G)\backslash E(T)$. Existiert kein solcher, beende den Algorithmus. Andernfalls augmentiere $\bar{C}_{T,e}$ größtmöglich zum maximalen Fluss $f_{i+1}$; jetzt existiert nach \cref{negKreis} eine Kante $e\neq e'\in E(C_{T,e})$ mit $f_{i+1}(e')=0$. Die neue zulässige Baumlösung ist $(T_{i+1}=T_i\backslash\{e'\}\cup\{e\},f_{i+1})$. Die verschiedenen Möglichkeiten zur Wahl von $e$ und $e'$ werden in \cref{ch:pivot} und \cref{ch:deg} näher beleuchtet.

Wir werden nun beweisen, dass der Algorithmus eine optimale Lösung des Transportproblems gefunden hat, wenn Phase 2 in Ermangelung einer geeigneten Kante $e$ beendet wird. \cref{iterierbar} ist etwas abstrakter gehalten, damit es noch mal in \cref{ch:alg2} Verwendung finden kann.

\begin{nota}Sei $N=(G,b,c,u)$ ein Netzwerk mit einem maximalen Fluss $f$, $T$ ein Wald von $G$ und $v,w\in V(G)$ zwei beliebige Knoten. Mit $v\xrightarrow{T}w$ bezeichnen wir den eindeutigen gerichteten Weg von $v$ nach $w$ in $R_{N,f}$, der nur über Kanten $e\in E(T)$ oder deren Residualkanten $\varphi(e)$ verläuft. Sollten $v$ und $w$ in $T$ nicht verbunden sein, entspricht der Weg dem leeren Graph. Insbesondere ist dieser Weg unabhängig von $f$ und für die Kosten gilt $c(w\xrightarrow{T}v) = -c(v\xrightarrow{T}w)$.\end{nota}

\begin{lem}\label{iterierbar}Sei $N=(G,b,c,u)$ ein Netzwerk mit einer zulässigen Baumlösung $(T,f)$ und $C$ ein ungerichteter Kreis in $G$, sodass für einen korrespondierenden gerichteten Kreis $\bar{C}$ in $R_{N,f}$ gilt:
\begin{enumerate}[(i)]
	\item $c(\bar{C})<0$\label{negativ}
	\item $\forall e\in E(C)\backslash E(T) \colon$ die korrespondierende Kante $\bar{e}\in E(\bar{C})$ darf vom Netzwerk-Simplex-Algorithmus mit zulässiger Baumlösung $(T,f)$ als eingehende Kante gewählt werden\label{wählbar}
\end{enumerate}
Dann existiert ein vom Algorithmus wählbares $\bar{e}\in\bar{C}$, sodass $c(\bar{C}_{T,\bar{e}})<0$.
\end{lem}
\begin{proof}Sei $N$ ein Netzwerk, $(T,f)$ eine zulässige Baumlösung und $C$ mit $\bar{C}$ zwei Kreise wie gefordert. Sei $E_{\bar{C},\neg T}\subseteq E(\bar{C})$ die Menge der korrespondierenden Kanten in $\bar{C}$ von den Nichtbaumkanten $E(C)\backslash E(T)$ und $E_{\bar{C},T}=E(\bar{C})\backslash E_{\bar{C},\neg T}$ dessen Komplement. Betrachten wir zunächst den Fall, dass $E_{\bar{C},T}=\emptyset$:

\begin{figure}[!ht]\centering
	\includestandalone{tikz_nontreecircle}
	\caption{Die durchgezogenen Kanten bilden den negativen Kreis $\bar{C}$. Jede Kante $(v,w)\in E(\bar{C})$ wird über $w\xrightarrow{T}v$ zu einem Kreis ergänzt.}
	\label{fig:NTC}
\end{figure}

Nach \cref{wählbar} gibt es für alle Kanten $e_1,\ldots,e_c\in E_{\bar{C},\neg T}=E(\bar{C})$ einen Kreis $\bar{C}_{T,e_i}$. Wie \cref{fig:NTC} veranschaulicht, können wir einen Kreis $\bar{C}_{T,e_i}$ erzeugen, indem wir $\bar{C}$ für alle $j\neq i$ um die Kreise $\bar{C}_{T,e_j}$ invertiert augmentieren. Damit gilt für die Kosten:

\begin{align}
c(\bar{C}_{T,e_i})=c(\bar{C})-\sum_{j\neq i} c(\bar{C}_{T,e_j})
\quad\Leftrightarrow\quad\sum_{i=1}^{c} c(\bar{C}_{T,e_i})=c(\bar{C})\label{eq:CircleCost}
\end{align}

Nach \cref{eq:CircleCost} muss mindestens ein Kreis $\bar{C}_{T,e_i}$ negative Kosten besitzen, da $c(\bar{C})<0$ ist. Mittels dieses Kreises ist die Aussage gezeigt.

Kommen wir zum Fall $E_{\bar{C},T}\neq\emptyset$. Da $T$ ein Baum ist, ist auch $E_{\bar{C},\neg T}\neq\emptyset$. Sollte $|E_{\bar{C},\neg T}|=1$ sein, ist $\bar{C}$ der gesuchte Kreis. Ansonsten werden wir $\bar{C}$ iterativ derart zu einem Kreis $\hat{C}$ verändern, dass $E_{\hat{C},\neg T}\subseteq E_{\bar{C},\neg T}$ und $|E_{\hat{C},\neg T}|=1$. Besitzt $\hat{C}$ negative Kosten, sind wir ebenfalls fertig, andernfalls werden wir zwischendurch bereits einen negativen Kreis $\bar{C}_{T,e}$ mit $e\in E_{\bar{C},\neg T}$ gefunden haben.

\begin{figure}[!ht]\centering
	\includestandalone{tikz_treecircleiter}
	\caption{Gestrichelte Linien sind zum Baum $T$ assoziiert. Der neue Kreis $C_1$ entsteht hier, indem wir von $x$ beginnend entgegen $\tilde{C}$ bis $z$ gehen und dann dem bisherigen Kreisverlauf $\bar{C}$ folgen.}
	\label{fig:TC}
\end{figure}

Betrachte einen Iterationsschritt mit negativen Kreis $\bar{C}$, sodass $|E_{\bar{C},\neg T}|>1$ gilt. Sei $e=(x,y)\in E_{\bar{C},\neg T}$ eine beliebige Nichtbaumkante. Wir betrachten nun den Kreis $\tilde{C}:=\bar{C}_{T,e}=y\xrightarrow{T}x\cup\{(x,y)\}$ und den Kantenzug $W:=x\xrightarrow{T}y\xrightarrow{\bar{C}-\{e\}}x$. Letzterer zerfällt bereinigt um in beide Richtungen begangene Kanten in kantendisjunkte Kreise $C_1,\ldots,C_c$ mit $c(W)=\sum_{i=1}^{c} c(C_i)$. \Obda sei die Kante $e$ in $C_1$ enthalten.

Sollten die Kosten $c(\tilde{C})<0$ negativ sein, ist $\tilde{C}$ nach \cref{wählbar} der gesuchte Kreis. Ist einer der Kreise $C_2,\ldots,C_c$ negativ, ersetzen wir $\bar{C}$ durch diesen Kreis. Andernfalls gilt für den Kreis $C_1$:
\begin{align*}
|E_{C_1,\neg T}|&<|E_{\bar{C},\neg T}| \\
c(C_1)&=c(\bar{C})-c(\tilde{C})-\sum_{i=2}^{c}c(C_i)\leq c(\bar{C})<0
\end{align*}

Damit können wir $\bar{C}:=C_1$ setzen. In beiden Fällen ist $|E_{\bar{C},\neg T}|$ im Vergleich zum Anfang der Iteration echt kleiner geworden. Wir iterieren weiter, bis $|E_{\bar{C},\neg T}|=1$ ist; dann ist $\hat{C}:=\bar{C}$ der gesuchte Kreis.\end{proof}

Um unser gewünschtes \cref{opt} zu zeigen, genügt es, einen negativen Kreis gemäß \cref{iterierbar} zu finden. Dafür werden wir den Begriff der Zirkulation einführen und den Zusammenhang zwischen maximalen Flüssen und Zirkulationen herstellen.

\begin{defn}\label{zirk}Sei $R_{N,f}= (\bar{G},\bar{f},b,\bar{c},\bar{u})$ ein Residualgraph. Eine \textbf{Zirkulation} auf $R_{N,f}$ ist eine Abbildung $z \colon E(\bar{G})\rightarrow\mathbb{N}_{\geq0}$, die folgende Eigenschaften erfüllt:
\begin{align}
&\text{(i)}&&\forall e\in E(\bar{G}) \colon z(e)\leq \bar{u}(e)\label{zirkI}\\
&\text{(ii)}&&\forall v\in V(\bar{G}) \colon \sum_{\mathclap{(w,v)\in E(\bar{G})}} z((w,v)) - \sum_{\mathclap{(v,w)\in E(\bar{G})}} z((v,w)) + b(v) = b(v)\label{zirkII}
\end{align}
Die \emph{Kosten} von $z$ betragen $c(z) = \sum\limits_{e\in E(\bar{G})} z(e)\cdot \bar{c}(e)$.
\end{defn}
\begin{anm}Zirkulationen sind an sich für beliebige gerichtete Graphen mit einer Kapazitätsfunktion definiert, für unsere Zwecke genügt die obige Version. Die Notation für den Graph aller durchflossenen Kanten überträgt sich.\end{anm}

\begin{lem}\label{zerl}Für jede Zirkulation $z$ gibt es eine endliche Menge von gerichteten Kreisen $\{C_1,\ldots,C_n\}$, sodass für Faktoren $\lambda_1,\ldots,\lambda_n\in\mathbb{N}$ die Kosten $c(z) = \sum_{i=1}^{n} \lambda_i\cdot c(C_i)$ zerfallen.\end{lem}
\begin{proof}Sei $R_{N,f}= (\bar{G},\bar{f},b,\bar{c},\bar{u})$ ein Residualgraph und $z$ eine Zirkulation. Wir führen eine Induktion über die Gesamtmenge Fluss $\sigma_z:=\sum_{e\in E(\bar{G})}z(e)$ der Zirkulation durch. Ist $\sigma_z=0$, so erfüllt die leere Menge das Lemma.
    
Sei $\sigma_z\neq0$ und damit gemäß \cref{zirk} $\sigma_z>0$. Damit ist $G_z$ nicht der leere Graph und nach \cref{zirkII} existiert ein gerichteter Kreis $C\subseteq G_z$. Wir definieren eine Abbildung $z'$ wie folgt:
\begin{equation*}z'(e)=\begin{cases}
z(e)-1&\text{wenn }e\in E(C)\\
z(e)&\text{sonst}\end{cases}
\end{equation*}
Offensichtlich ist $z'$ eine Zirkulation und $\sigma_{z'}<\sigma_z$. Durch die Induktionsbehauptung bekommen wir für $z'$ eine Menge $\mathscr{C'}:=\{C'_1,\ldots,C'_n\}$ mit Faktoren $\lambda'_1,\ldots,\lambda'_n$, sodass $c(z') = \sum_{i=1}^{n} \lambda'_i\cdot c(C'_i)$.

Wir wandeln jetzt $\mathscr{C}:=\mathscr{C'}$ und die dazugehörigen Faktoren $\lambda_i:=\lambda'_i$ so ab, dass die Eigenschaft für $z$ erfüllt ist. Sollte $C=C_j\in\mathscr{C}$ sein, erhöhen wir den Wert $\lambda_j$ um $1$, andernfalls fügen wir $C$ zu $\mathscr{C}$ hinzu und ergänzen den Faktor $\lambda_{n+1}:=1$. In beiden Fällen gilt
\begin{equation*}c(z)=c(C)+c(z')=c(C)+\sum_{i=1}^{n} \lambda'_i\cdot c(C'_i)=\sum_{i=1}^{|\mathscr{C}|} \lambda_i\cdot c(C_i)\end{equation*}
nach Konstruktion. Damit ist die Aussage für alle Zirkulationen gezeigt.\end{proof}

\begin{nota}Seien $\hat{f}$ und $f$ zwei maximale Flüsse auf einem Netzwerk $(G,b,c,u)$. Mit $\hat{f}-f$ sei die folgendermaßen auf dem Residualgraphen $R_{N,f}$ definierte Abbildung $z\colon E(\bar{G})\rightarrow\mathbb{N}_{\geq0}$ gemeint:
\begin{equation*}z(e)=\begin{cases}
\max\{\hat{f}(e)-f(e),0\}&\text{wenn } e\in E(G)\\
\max\{f(\varphi(e))-\hat{f}(\varphi(e)),0\}&\text{wenn } e\in \varphi(E(G))\end{cases}
\end{equation*}\end{nota}

\begin{lem}Seien $\hat{f}$ und $f$ zwei maximale Flüsse auf einem Netzwerk $(G,b,c,u)$. Dann ist $z:=\hat{f}-f$ eine Zirkulation auf $R_{N,f}$ und für die Kosten gilt $c(z)=c(\hat{f})-c(f)$.\end{lem}
\begin{proof}Seien $E:=E(G)$ alle Kanten des Graphen und $\bar{E}:=\varphi(E)$ ihre Residualkanten in $R_{N,f}$. Zunächst betrachten wir die Kosten von $z$:
\begin{align*}c(z)&=\sum_{\mathclap{e\in E\amalg\bar{E}}} z(e)\cdot \bar{c}(e)\\
&=\sum_{e\in E}\max\{\hat{f}(e)-f(e),0\}\cdot c(e) + \sum_{\bar{e}\in \bar{E}}\max\{f(\varphi(e))-\hat{f}(\varphi(e)),0\}\cdot c_f(\bar{e})\\
&=\sum_{e\in E}\max\{\hat{f}(e)-f(e),0\}\cdot c(e) + \sum_{\bar{e}\in \bar{E}}-\min\{\hat{f}(\varphi(e))-f(\varphi(e)),0\}\cdot(- c(\varphi(\bar{e})))\\
&=\sum_{e\in E}\max\{\hat{f}(e)-f(e),0\}\cdot c(e) + \sum_{e\in E}\min\{\hat{f}(e)-f(e),0\}\cdot c(e)\\
&=\sum_{e\in E} (\hat{f}(e)-f(e))\cdot c(e) = c(\hat{f})-c(f)
\end{align*}
Diese Teilaussage gilt also. Nun überprüfen wir, ob $z$ alle Eigenschaften wie in \cref{zirk} gefordert erfüllt. Für alle Kanten $e\in E$ ist
\begin{equation*}z(e)=\max\{\hat{f}(e)-f(e),0\}\leq\hat{f}(e)\leq u(e)=\bar{u}(e),\end{equation*}
für Residualkanten $\bar{e}\in \bar{E}$ gilt
\begin{equation*}z(\bar{e})=\max\{f(\varphi(\bar{e}))-\hat{f}(\varphi(\bar{e})),0\}\leq f(\varphi(\bar{e}))\leq u_f(\bar{e})\leq \bar{u}(e).\end{equation*}
\cref{zirkI} ist also erfüllt.

Damit bleibt zu zeigen, dass sich der Fluss von $z$ für alle Knoten $v\in V(\bar{G})$ gemäß \cref{zirkII} ausgleicht:
\begin{align*}&\sum_{\mathclap{(w,v)\in E\amalg\bar{E}}} z((w,v)) - \sum_{\mathclap{(v,w)\in E\amalg\bar{E}}} z((v,w)) + b(v)\\
={}&\sum_{\mathclap{(w,v)\in E}} \Big(\hat{f}((w,v)) -  f((w,v))\Big) -  \sum_{\mathclap{(v,w)\in E}} \Big(\hat{f}((v,w)) - f((v,w))\Big) + b(v)\\
={}&b(v)-b(v)+b(v)=b(v)\end{align*}

Die erste Gleichheit gilt gemäß Konstruktion von $z$, da ein theoretischer negativer Fluss auf einer Kante $e\in E(G)$ als positiver auf der Residualkante $\varphi(e)$ umgesetzt wird. Die zweite Gleichheit gilt gemäß \cref{DefMaxFlowII} aus \cref{DefMaxFlow}. Damit ist \cref{zirkII} erfüllt und $z$ eine Zirkulation.\end{proof}

Zur Veranschaulichung eines Beweises mittels Zirkulation zeigen wir erneut \cref{TreeUnique}, bevor wir zu \cref{opt} übergehen. 

\TreeUnique*
\begin{proof}Seien $(T,f)$ und $(T,f')$ zwei zulässige Baumlösungen über demselben aufspannenden Baum $T$ des Netzwerkes $N=(G,b,c,u)$ und $z:=f'-f$ eine Zirkulation auf $R_{N,f}$. Da $G_f\cup G_{f'}\subseteq T$ ein Wald ist, gilt dies auch für $\bar{G}_z$. Damit gilt $z(\bar{e}) = 0$ für alle Kanten $\bar{e}\in E(\bar{G})$, womit schon $f=f'$ war.\end{proof}

\begin{thm}\label{opt}Sei $N$ eine Instanz des Transportproblems mit einer zulässigen Baumlösung $(T,f)$. Existiert kein negativer Kreis $\bar{C}_{T,e}$, so ist $f$ eine optimale Lösung.\end{thm}
\begin{proof}Wir werden zeigen, dass ein negativer Kreis $\bar{C}_{T,e}$ existiert, wenn die betrachtete zulässige Baumlösung $(T,f)$ nicht optimal ist. Dazu finden wir einen gerichteten negativen Kreis, für den wir \cref{iterierbar} anwenden können.
	
Sei $N=(G,b,c,u)$ ein Netzwerk mit einer optimalen zulässigen Baumlösung $(\hat{T},\hat{f})$ und einer zulässigen Baumlösung $(T,f)$, sodass $c(f)>c(\hat{f})$. Nach \cref{zerl} für die Zirkulation $z:=\hat{f}-f$ sind ihre Kosten $c(z)=c(\hat{f})-c(f)<0$ negativ und zerfallen für eine Menge $\mathscr{C}=\{C_1,\ldots,C_n\}$ mit Faktoren $\lambda_1,\ldots,\lambda_n$. \Obda seien alle $\lambda_i\neq 0$.

Da $\sum_{C_i\in\mathscr{C}} \lambda_i\cdot c(C_i)=c(z)<0$ ist, gibt es mindestens einen negativen Kreis $C_j\in\mathscr{C}$. Sei $e\in E(C_j)$ eine Kante dieses Kreises. Ist $e$ eine Residualkante, so gilt $\hat{f}(\varphi(e))-f(\varphi(e))<0$, woraus $f(e)>0$ und damit $\varphi(e)\in E(T)$ folgt. Ansonsten ist $e\in E(G)$ und vom Algorithmus wählbar. Wir können also \cref{iterierbar} anwenden und sind fertig.\end{proof}

\begin{kor}\label{korrekt}Sei $N$ eine Instanz des Transportproblems mit einer zulässigen Initialbaumlösung $(T_0,f_0)$. Determiniert der Netzwerk-Simplex-Algorithmus, so ist er korrekt.\qed\end{kor}

Bislang haben wir nicht gezeigt, dass der Algorithmus immer terminiert. Dies wäre offensichtlich, wenn bei jeder Iteration von $(T,f)$ zu $(T',f')$ die Kosten sinken würden, also $c(f')<c(f)$ wäre. Es gibt jedoch sogenannte \emph{degenerierte Iterationen}, in denen ein negativer Kreis $\bar{C}_{T,e}$ um $\delta=0$ augmentiert wird. Entfernt der Algorithmus danach eine Kante $e\neq e'\in E(C_{T,e})$, verändert sich nur der Baum. Der nächste Abschnitt beschäftigt sich damit, wie sichergestellt werden kann, dass der Algorithmus zumindest jeden Baum höchstens einmal betrachtet.

\subsection{Degenerierte Iterationen}\label{ch:deg}

\begin{defn}Wird in einer Iteration der Phase 2 des Netzwerk-Simplex-Algorithmus ein negativer Kreis $\bar{C}_{T,e}$ um $\delta=0$ augmentiert, so bezeichnen wir dies als \textbf{degenerierte Iteration}.
\end{defn}

\begin{figure}[!ht]\centering
	\includestandalone{tikz_deg}
	\caption{Die Kanten des Baumes $T_i$ sind gestrichelt, der gewählte negative Kreis ist jeweils eindeutig. Die Startlösung kann erst nach einer degenerierten Iteration verbessert werden.}
	\label{fig:deg}
\end{figure}

Degenerierte Iterationen entstehen, wenn bei einer zulässigen Baumlösung $(T,f)$ nicht alle Kanten von $T$ Fluss aufweisen, also $T\neq T_f$ ist. $(T,f)$ wird dann auch als \emph{degeneriert} bezeichnet. In einer ungünstigen Konstellation von der deterministischen Wahl der hinzugefügten Kante $e$ und entfernten Kante $e'$ kann es zum \emph{Cycling} kommen, also zu einem Kreisschluss von Bäumen, die wiederholt iteriert werden. Dies tritt sehr selten auf, für ein künstlich konstruiertes Beispiel siehe \cite[S. 303]{NSAbook}.

\cite[Cunningham, 1976]{cycling} führte eine Methode ein, mit der Cycling verhindert werden kann, ohne dass die Wahl der hinzugefügten Kante $e$ eingeschränkt wird. Dafür benötigen wir folgende Definition:

\begin{defn}Sei $N=(G,b,c,u)$ eine Instanz des Transportproblems. Ein aufspannender Baum $T$ von $G$, ein maximaler Fluss $f$ auf $N$ und ein Wurzelknoten $r\in V(T)=V(G)$ bilden eine \textbf{stark zulässige Baumlösung} $(T,f)_r$, wenn $(T,f)$ eine zulässige Baumlösung ist und zusätzlich jede Kante $e=(v,w)\in E(T)$ mit $f(e)=0$ von der Wurzel weggerichtet, also $e$ im Weg $r\xrightarrow{T}w$ enthalten ist.\end{defn}

TODO: An dieser Stelle folgt ein Absatz, der erklärt, wie genau die ausgehende Kante abhängig vom Kreis und dem Wurzelknoten gewählt werden muss. Es folgt das Lemma, dass dann der Algorithmus terminiert, für den Beweis wird auf die Literatur verwiesen. Dieser Teil fehlt momentan noch, da ich das noch mit meinem programmierten Algorithmus abgleichen muss. Außerdem stimmten die Informationen im nächsten Absatz nicht bzw. sind unvollständig.

Die Laufzeit des Netzwerk-Simplex-Algorithmus ist bislang ungeklärt. Für bestimmte Varianten wurden exponentielle Instanzen gefunden, die auf \emph{Stalling} basieren, also einer exponentiellen Anzahl degenerierter Iterationen. Mit diesen werden wir uns in \cref{ch:lit} näher befassen. Meine eigene, experimentelle Suche nach schlechten Instanzen findet sich in \cref{ch:erg}. Zunächst vervollständigen wir den Algorithmus um die Wahl des negativen Kreises $\bar{C}_{T,e}$ und die Erzeugung einer initialen Baumlösung.

\subsection{Pivotalgorithmen}\label{ch:pivot}
Sei $N=(G,b,c,u)$ ein Netzwerk mit zulässiger Baumlösung $(T,f)$. Algorithmen, die aus der Menge $\bar{C}_T=\{\bar{C}_{T,e}\mid c(\bar{C}_{T,e})<0)\}$ aller möglichen Iterationen eine auswählen, heißen \emph{Pivotalgorithmen}. In der Praxis wird der Pivotalgorithmus nur auf einer Teilmenge von $\bar{C}_T$ ausgeführt, um Rechenzeit zu sparen. In dieser Bachelorarbeit werden nur drei naheliegende, einfache Pivotalgorithmen betrachtet.

\subsubsection{Maximum Value}
Der erste Ansatz ist es, den negativsten Kreis zu wählen, sprich \begin{equation*}
\tilde{C}:=\argmin_{\bar{C}_{T,e}\in\bar{C}_T} \{c(\bar{C}_{T,e})\}
\end{equation*}

Diesen Weg werden wir mit \emph{MaxVal} bezeichnen.

\subsubsection{Maximum Revenue}
Die Kostenverringerung nach der Augmentierung beträgt $\delta\cdot c(\bar{C}_{T,e})$, ist also von $\delta$ abhängt. Der Pivotalgorithmus \emph{MaxRev} maximiert diesen Wert:
\begin{flalign*}
&&\tilde{C}:=\argmin_{\bar{C}_{T,e}\in\bar{C}_T} \{\delta_e\cdot c(\bar{C}_{T,e})\}
&&\delta_e:=\min_{e'\in E(\bar{C}_{T,e})}\{\bar{u}(e')-\bar{f}(e')\}
\end{flalign*}

Nach \cref{negKreis} ist jedes $\delta_e$ endlich. Sollten nur degenerierte Iterationen zur Auswahl stehen, ist $c(\tilde{C})=0$. In dem Fall wendet meine konkrete Implementierung MaxVal an; hier sind aber auch andere Strategien denkbar.

\subsubsection{Random}
Ein überraschend effektiver Ansatz ist es, $\tilde{C}\in \bar{C}_T$ zufällig zu wählen. Gerade für diesen mit \emph{Random} bezeichneten Weg ist es schwierig, untere oder obere Schranken zu beweisen.

\subsection{Initialisierung} \label{ch:init}
Der letzte verbleibende Schritt, um einen vollständigen Algorithmus zur Lösung des Transportproblems zu erlangen, ist das Finden einer initialen, stark zulässigen Baumlösung $(T_0,f_0)_r$.

Hierfür wenden wir einen Trick an: Sei $N=(G,b,c,u)$ eine Instanz des Transportproblems. Wir fügen dem Netzwerk einen zusätzlichen Transitknoten $a$ mit $b(a)=0$ hinzu, der gemeinhin als künstlicher \textit{(artificial)} Knoten bezeichnet wird. Für alle Quellen $v\in V(G)\colon b(v)>0$ ergänzen wir eine künstliche Kante $(v,a)$, für alle Senken und Transitknoten $a\neq w\in V(G)\colon b(w)\leq0$ eine künstliche Kante $(a,w)$. Sei $G'$ der entstehende Graph.

Sämtliche künstlichen Kanten $e_v$ haben eine unbegrenzte Kapazität, die Kosten sind ebenfalls unendlich und können mit $c(e_v)=|V(G)|\cdot\max_{e\in E(G)}\{c(e)\}+1$ abgeschätzt werden, da jeder Weg in $G$ geringere Kosten aufweist. Leicht finden wir nun die stark zulässige Baumlösung $(T_0,f_0)_a$, wobei $T_0=(V(G),\{e_v\mid v\in V(G)\})$ ist. Nach \cref{TreeUnique} ist $f_0$ eindeutig. Diese Art der Initialisierung werden wir im Folgenden mit \emph{HC} für \emph{High-Cost-Initialisierung} abkürzen.

\begin{lem}Sei $N=(G,b,c,u)$ eine Instanz des Transportproblems und $(T',f)$ die Lösung des Netzwerk-Simplex-Algorithmus mit \emph{HC} auf dem erweiterten Graphen $G'$. Die Instanz $N$ ist genau dann lösbar, wenn $G'_f$ keine künstlichen Kanten enthält.\end{lem}
\begin{proof}\label{solvable}\mbox{}
\begin{itemize}[topsep=0pt]
	\item[\enquote{$\Rightarrow$}] Sei $\hat{f}$ ein maximaler Fluss von $N$ und $(T',f)$ die Lösung des Algorithmus. $\hat{f}$ ist auch ein maximaler Fluss des Netzwerkes $(G',b,c,u)$; damit gilt nach \cref{BLex} und \cref{korrekt} $c(f)\leq c(\hat{f})<\infty$. Da die Kosten einer künstlichen Kante unendlich sind, kann keine davon in $G'_f$ enthalten sein.
	
	\item[\enquote{$\Leftarrow$}] Sei $(T',f)$ die Lösung des Algorithmus. Wenn $G'_f$ keine künstlichen Kanten enthält, ist $f_{|E(G)}$ ein maximaler Fluss von $N$, womit diese Instanz wiederum lösbar ist.\qedhere
\end{itemize}
\end{proof}

Sollen Instanzen mit vielen Knoten oder hohen Kosten der teuersten Kante gelöst werden, so kann es passieren, dass die Kosten der künstlichen Kanten den darstellbaren Bereich der gängigen Datentypen sprengen. Unter anderem aus diesem Grund gibt es die \emph{Low-Cost-Initialisierung}, bei uns kurz \emph{LC}.

Bei dieser wird die gegebene Instanz $N=(G,b,c,u)$ des Transportproblems zunächst zu $N'=(G,b,c',u)$ abgewandelt. Dabei ist $c'$ die Nullfunktion, also $c(e)=0$ für jede Kante $e\in E(G)$. Für $N'$ verwenden wir den Netzwerk-Simplex-Algorithmus mit \emph{HC} und erhalten eine stark zulässige Baumlösung $(T',f')_a$, wobei die Kosten einer künstlichen Kante $c(e_v)=|V(G)|\cdot0+1=1$ betragen.

Offensichtlicherweise ist $N$ genau dann lösbar, wenn $N'$ lösbar ist. Wenn $c(f')>0$ ist, so ist $N$ nicht lösbar und wir sind fertig. Andernfalls iterieren wir $(T',f')_a$ degeneriert, bis $a$ ein Blatt von $T'$ ist. Dies wird in \cref{HCLC} detailliert erläutert.

Damit bekommen wir die zulässige Baumlösung $(T=T'-a, f=f'_{|E(G)})$ für $N'$ und damit für $N$. Als neuen Wurzelknoten können wir den eindeutigen Nachbar $k_a$ von $a$ in $T'$ wählen; damit bekommen wir die stark zulässige Baumlösung $(T,f)_{k_a}$.

\section{Erweiterung auf beschränkte Kapazitäten}\label{ch:alg2}
Wir werden nun den Netzwerk-Simplex-Algorithmus auf das Min-Cost-Flow-Problem erweitern. Die Funktionsweise bleibt dieselbe, nur die betrachteten zulässigen Baumlösungen verändern sich leicht:

\begin{defn}Sei $N=(G,b,c,u)$ ein Netzwerk. Ein aufspannender Baum $T$ von $G$ und ein maximaler Fluss $f$ auf $N$ bilden eine \textbf{zulässige Baumlösung} $(T,f)$, wenn für alle Kanten $e\in E(G)\backslash E(T)$ außerhalb des Baums $f(e) = 0$ oder $f(e)=u(e)$ gilt.\end{defn}
\begin{anm}Kanten, die maximal durchflossen sind, bezeichnen wir als \emph{saturiert}.\end{anm}

\begin{defn}Sei $N=(G,b,c,u)$ ein Netzwerk. Ein aufspannender Baum $T$ von $G$, ein maximaler Fluss $f$ auf $N$ und ein Wurzelknoten $r\in V(T)=V(G)$ bilden eine \textbf{stark zulässige Baumlösung} $(T,f)_r$, wenn $(T,f)$ eine zulässige Baumlösung ist und folgendes gilt:
\begin{alignat*}{2}
\forall e=(v,w)\in E(T)\colon&(f(e)=0 \Rightarrow e\in E(r\xrightarrow{T}w))&\land\\
&(f(e)=u(e) \Rightarrow e\in E(w\xrightarrow{T}r))
\end{alignat*}\end{defn}
\begin{anm}An dieser Stelle sollten sämtliche Kanten $e\in E(G)$ mit $u(e)=0$ aus dem Graphen entfernt werden; sie sind für das Problem ohnehin irrelevant.\end{anm}

\begin{figure}[!ht]\centering
	\includestandalone{tikz_NSA}	
	\caption{Derselbe, durch gestrichelte Linien dargestellte Baum mit unterschiedlichen maximalen Flüssen. Der Wurzelknoten ist die Senke.}
	\label{fig:NSA}
\end{figure}

Im Falle beschränkter Kapazitäten erlauben wir für eine zulässige Baumlösung $(T,f)$ auch Fluss außerhalb von $T$, insofern er die Kapazität der Kante voll ausnutzt. \cref{fig:NSA} veranschaulicht, warum dies notwendig ist. Solche saturierten Kanten $e\in E(G)\backslash E(T)$ sind als eingehende Kante keine sinnvolle Wahl mehr, stattdessen kann der Algorithmus über den Kreis $\bar{C}_{T,\bar{e}}$ der Residualkante $\bar{e}$ augmentieren, sofern er negativ ist. Für stark zulässige Baumlösungen müssen nun zusätzlich alle saturierten Kanten des Baumes zum Wurzelknoten hingerichtet sein.

Die in \cref{TreeUnique} bewiesene Eindeutigkeit von $f$ gilt nun nicht mehr, wie \cref{fig:NSA} veranschaulicht. Wir haben diese Eigenschaft bei der Initialisierung genutzt. Der Leser kann sich an dieser Stelle davon überzeugen, dass diese trotzdem komplett identisch durchführbar ist.

Wir werden jetzt \cref{negKreis,BLex,opt,TP} auf das Min-Cost-Flow-Problem erweitern.

\begin{lem}\label{negKreis2}Sei $N$ ein Netzwerk, $R_{N,f}$ sein Residualgraph und $C$ ein gerichteter Kreis in $R_{N,f}$ mit negativen Kosten. Der größte Wert $\delta$, um den $C$ augmentiert werden kann, ist endlich, und nach der Augmentierung um $\delta$ zum neuen maximalen Fluss $f'$ existiert eine Residualkante $\bar{e}\in E(C)$, sodass die korrespondierende Kante einen Fluss von $f'(\varphi(e)))=0$ besitzt, oder eine Kante $e\in E(C)\cap E(G)$ mit $f'(e)=u(e)$.\end{lem}
\begin{proof}Sei $N=(G,b,c,u)$ ein Netzwerk mit maximalen Fluss $f$ und $C$ ein negativer Kreis in $R_{N,f}$. Sei $\tilde{e}:=\argmin_{e\in E(C)}\{\bar{u}(e)-\bar{f}(e)\}$ und $\delta:=\bar{u}(\tilde{e})-\bar{f}(\tilde{e})$. Analog zum Beweis von \cref{negKreis} ist $\delta$ endlich.
	
Ist $\tilde{e}$ eine Residualkante, so besitzt $e:=\varphi(e)$ einen Fluss von
\begin{equation*}
f'(e)=f(e)-\delta=f(e)-u_f(\bar{e})=f(e)-f(e)=0.\end{equation*} Andernfalls ist $\tilde{e}\in E(C)\cap E(G)$ mit einem Fluss von
\begin{equation*}
f'(\tilde{e})=f(\tilde{e})+\delta=f(\tilde{e})+\bar{u}(\tilde{e})-\bar{f}(\tilde{e})=\bar{u}(\tilde{e})=u(\tilde{e}).\qedhere\end{equation*}\end{proof}

\begin{nota}Sei $f$ ein maximaler Fluss für ein Netzwerk $(G=(V,E),b,c,u)$ und $H=(V'\subseteq V, E'\subseteq E)$ ein Teilgraph. Mit $H^f=H_f\backslash\{e\in E'\mid f(e)=u(e)\}$ bezeichnen wir den Graph aller durchflossenen, aber nicht saturierten Kanten.\end{nota}

\begin{thm}\label{BLex2}Sei $N$ ein Netzwerk mit einem maximalen Fluss $f$. Es existiert ein maximaler Fluss $\hat{f}$, sodass $c(\hat{f})\leq c(f)$ ist und eine zulässige Baumlösung $(\hat{T},\hat{f})$ existiert.\end{thm}
\begin{proof}Sei $N=(G,b,c,u)$ ein Netzwerk mit maximalen Fluss $f$. Diesmal werden wir $f$ zu einem maximalen Fluss $\hat{f}$ umwandeln, sodass $G^{\hat{f}}$ ein Wald ist. Der Beweis verläuft nun komplett analog zum Beweis von \cref{BLex}, statt \cref{negKreis} benutzen wir \cref{negKreis2}.\end{proof}

\begin{kor}Für jede lösbare Instanz des Min-Cost-Flow-Problems existiert eine zulässige Baumlösung $(T,f)$, sodass der maximale Fluss $f$ minimale Kosten hat.\qed\end{kor}

\begin{thm}\label{opt2}Sei $N$ ein Netzwerk mit einer zulässigen Baumlösung $(T,f)$. Existiert kein negativer Kreis $\bar{C}_{T,e}$, so ist $f$ eine optimale Lösung.\end{thm}
\begin{proof}Wir werden wiederum zeigen, dass ein negativer Kreis $\bar{C}_{T,e}$ existiert, wenn die betrachtete Baumlösung nicht optimal ist.

Sei $N=(G,b,c,u)$ ein Netzwerk mit einer optimalen zulässigen Baumlösung $(\hat{T},\hat{f})$ und einer zulässigen Baumlösung $(T,f)$, sodass $c(f)>c(\hat{f})$ ist. Wieder betrachten wir die Zirkulation $z:=\hat{f}-f$, ihre Zerlegung in eine Menge $\mathscr{C}$ gerichteter Kreise mit Faktoren $\lambda_1,\ldots,\lambda_n$ und den negativen Kreis $C_j\in\mathscr{C}$.

Sei $e\in E(C_j)$ eine Kante dieses Kreises. Ist $e$ eine Residualkante, so gilt wie gehabt $\hat{f}(\varphi(e))-f(\varphi(e))<0$, woraus $f(e)>0$ folgt. Damit ist $\varphi(e)\in E(T)$ oder $\varphi(e)$ saturiert. Andernfalls ist $e\in E(G)$ und $f(e)<\hat{f}(e)$, womit $e$ nicht saturiert sein kann. Die Bedingungen für \cref{iterierbar} sind erneut gegeben.\end{proof}

\begin{kor}Der Netzwerk-Simplex-Algorithmus mit einer Initialisierung nach \cref{ch:init}, einem beliebigen Pivotalgorithmus und gemäß \cref{ch:deg} auf stark zulässige Baumlösungen beschränkt löst das Min-Cost-Flow-Problem korrekt.\qed\end{kor}
\chapter{Implementierung} \label{ch:prog}
Der Programmierteil dieser Bachelorarbeit implementiert den Netzwerk-Simplex-Algorithmus wie in \cref{ch:NSA} beschrieben, wobei einzelne Strukturen verändert wurden. Der Code ist in \cpp  geschrieben und hält sich an den \cppelf-Standard. Es folgt eine vollständige Übersicht der Codebestandteile, für das Gesamtkonstrukt siehe TODO Anhang. An einigen Stellen wurden für ein besseres Layout Variablennamen angepasst, Kommentare gekürzt und Funktionsdeklarationen weggelassen.

\section{Netzwerke}
Zur Umsetzung der Graphenstruktur habe ich eine Klasse \lstinline|class Network| geschrieben, die auf \lstinline|struct Node| sowie \lstinline|class Edge| basiert und den Fluss händelt. Damit kann \lstinline|Network| beispielsweise sicherstellen, dass nie Kanten entfernt werden, die aktuell durchflossen sind.

\subsection{Knoten}
\lstinline|struct Node| hat außer einem Konstruktor keinerlei Funktionen, weswegen es nicht in einer eigenen Klasse ausgelagert, sondern in \lstinline|network.h| enthalten ist. 

\begin{lstlisting}
struct Node {
  size_t id;
  intmax_t b_value;
  //defined by (the other) nodeID
  //always incoming and outgoing due to residual edges
  std::set<size_t> neighbours;

  Node (size_t _id, intmax_t _b_value)
    : id(_id), b_value(_b_value) {};
};
\end{lstlisting}

Ein Knoten wird eindeutig durch seine \lstinline|size_t id| identifiziert, weitere Eigenschaften sind sein b-Wert aus $\mathbb{Z}$ und seine Nachbarknoten. Letztere Menge differenziert nicht zwischen Nachbarn durch eingehende und ausgehende Kanten, da diese Unterscheidung im Residualgraph entfällt.

\subsection{Kanten}
Die Klasse \lstinline|class Edge| enthält alle Informationen über Kanten: Kosten, Kapazität, Fluss und die jeweiligen Endknoten. Residualkanten werden durch das Flag \lstinline|isResidual| gekennzeichnet. An dieser Stelle fällt auf, dass die Kapazität nicht nachträglich verändert werden kann, obwohl dies für Residualkanten notwendig wäre. Ich habe mich in meiner Implementierung dafür entschieden, auf Residualkanten Fluss zu erlauben und damit eine veränderliche Kapazität zu simulieren. Dies wird in \cref{ch:graph} näher erläutert.

\begin{lstlisting}
class Edge {
public:
  //edges are initialized with a flow of zero
  Edge (size_t node0, size_t node1, intmax_t cost,
        intmax_t capacity, bool isResidual = false);

  //returns true when flow change was successful
  bool changeFlow (intmax_t value);
  bool changeFlowPossible (intmax_t value);
  //toggles cost betweeen itself and 0
  void toggleCost ();

private:
  size_t node0, node1;
  intmax_t cost, capacity, flow, toggledCost;
  bool isResidual, isToggled = false;
};
\end{lstlisting}

Die Funktion \lstinline|changeFlow| stellt über die Hilfsfunktion \lstinline|changeFlowPossible| sicher, dass bei einer Flussveränderung keine Kapazitätsschranken verletzt werden. Mittels \lstinline|toggleCost| können die Kosten einer Kante zwischen Null und ihrem eigentlichen Wert variiert werden. Somit lässt sich leicht die in \cref{ch:init} beschriebene Low-Cost-Initialisierung umsetzen, siehe dazu \cref{ch:HCLC}.

\subsection{Graph}\label{ch:graph}
Werfen wir einen Blick darauf, wie die Knoten und Kanten in einem Netzwerk gespeichert werden. Ich habe mich dafür entschieden, die Objekte für eine logarithmische Laufzeit von \lstinline|find|, \lstinline|insert| sowie \lstinline|delete| in einer \lstinline|std::map| abzulegen. Die Knoten werden durch ihre \lstinline|size_t id| identifiziert, Kanten wiederum durch Anfangsknoten, Endknoten und die \lstinline|bool isResidual|. Eine \lstinline|std::map| benötigt einen Vergleichsoperator, daher habe ich für die Kanten \lstinline|custComp| definiert, der nach \lstinline|size_t node0|, \lstinline|size_t node1| und zuletzt \lstinline|bool isResidual| vergleicht.

\begin{lstlisting}
class Network {
private:
  intmax_t flow = 0, cost = 0;
  std::vector<size_t> sources, sinks, transit;

  std::map<std::tuple<size_t, size_t, bool>,
           Edge, custComp> edges;
  std::map<size_t, Node, std::less<size_t>> nodes;
};
\end{lstlisting}

Ein Netzwerk unterteilt zusätzlich seine Knoten nach ihrem b-Wert abzüglich des aktuellen Flusses in Quellen, Senken und Transitknoten. Eine Instanz mit einem maximalen Fluss besteht dementsprechend nur aus Transitknoten. Es kennt außerdem seinen \lstinline|intmax_t flow|, also die Anzahl an Fluss, die von den Quellen zu den Senken transportiert wird, und die dafür anfallenden Kosten \lstinline|intmax_t cost|. Für den Algorithmus ist ersteres nicht relevant, da wir nur auf maximalen Flüssen arbeiten.

Der Graph $G$ mit einem Fluss $f$ und der dazugehörige Residualgraph $\bar{G}$ werden nicht als zwei verschiedene Objekte der Netzwerk-Klasse geführt, sondern in einer Instanz gemeinsam gespeichert. Um dies zu bewerkstelligen, wird bei jeder Veränderung an einer Kante gleichzeitig die eindeutige Residualkante aktualisiert. Zur Veranschaulichung folgt der Quellcode von \lstinline|Network::addEdge| ohne die Prüfung der Zulässigkeit.

\begin{lstlisting}
bool Network::addEdge(Edge e) {
  std::tuple<size_t, size_t, bool> key
    = std::make_tuple(e.node0, e.node1, e.isResidual);

  //insert both the edge and the residual edge
  edges.insert(std::make_pair(key, e));
  e.invert();
  edges.insert(std::make_pair(invertKey(key), e));

  nodes.find(node0)->second.neighbours.insert(node1);
  nodes.find(node1)->second.neighbours.insert(node0);
  return true;
}
\end{lstlisting}

Dabei vertauscht \lstinline|invertKey| die beiden Knoten und negiert \lstinline|isResidual|. Die oben nicht erwähnte Funktion \lstinline|Edge::invert| spiegelt die Veränderungen von \lstinline|invertKey| und negiert die Kosten gemäß \cref{defRes}. Die Kapazität wird beibehalten, stattdessen wird der Fluss auf \lstinline|flow = capacity - flow;| gesetzt. Der größte Wert, um den die Residualkante $\bar{e}$ augmentiert werden kann, ist also gleich $f(e)$. Sofern wir $f$ auf Nicht-Residualkanten einschränken, ist diese Implementierung äquivalent zur Definition. Dieser alternative Ansatz fordert noch ein wenig Aufmerksamkeit bei der Klassenfunktion \lstinline|Network::changeFlow|,\footnotemark{} danach können wir bei der Umsetzung des Algorithmus größtenteils ignorieren, ob eine Kante residual ist.

\footnotetext{Für alle Kanten $e$ des Kreises wenden wir \lstinline|e.changeFlow(f)| sowie \lstinline|e.invert().changeFlow(-f)| an und aktualisieren die Kosten auf \lstinline|this->cost += e.cost*f;|.}

Das Netzwerk wird leer oder mit einer Anzahl \lstinline|size_t noOfNodes| von Transitknoten initialisiert und durch Hinzufügen von Knoten und Kanten vervollständigt. Sofern keine Knoten gelöscht werden, ist die \lstinline|size_t id| bei null beginnend lückenlos aufsteigend.

\begin{lstlisting}
class Network {
public:
  Network(size_t noOfNodes);

  bool addEdge(Edge e);
  //returns nodeID
  size_t addNode(intmax_t b_value = 0);
  
  //fails and returns 0 if there's flow left
  bool deleteEdge(size_t node0, size_t node1);
  //fails if there's flow on an edge to this node left
  bool deleteNode(size_t nodeID);

  //takes a path from a source to a sink
  //doesn't use residual edges
  bool addFlow(std::vector<size_t>& path, intmax_t flow);
  bool changeFlow(Circle& c, intmax_t flow);
};
\end{lstlisting}

Die Funktionen \lstinline|deleteEdge| und \lstinline|deleteNode| schlagen fehl, wenn auf den zu löschenden Kanten noch Fluss ist. So kann unter am Ende des Algorithmus durch Löschen des künstlichen Knotens überprüft werden, ob die Instanz lösbar ist. Für Phase 1 des Algorithmus (siehe \cref{ch:HCLC}) nutzen wir die Funktion \lstinline|addFlow|, die einen Weg von einer Quelle zu einer Senke verlangt.

\lstinline|changeFlow| wird in jeder Iteration der zweiten Phase des Netzwerk-Simplex-Algorithmus aufgerufen und augmentiert den übergebenen Kreis \lstinline|Circle c| um einen beliebigen Wert, sofern dies möglich ist. Dieser Wert wird analog zu $\delta$ aus \cref{ch:NSA} der größte zulässige sein. Die hier genutzte \lstinline|class Circle| werden wir uns nun genauer anschauen.

\section{Die Klasse \emph{Circle}}
Sei $G$ ein gerichteter Graph mit $n$ Knoten sowie $m$ Kanten und $T$ ein aufspannender Baum von $G$. In der Herleitung des Algorithmus haben wir genutzt, dass für eine Kante $e\in E(G)\backslash E(T)$ außerhalb des Baumes der Kreis $C_{T,e}$ eindeutig ist. Wird $T$ im Speicher gehalten, kann beispielsweise durch Tiefensuche in Laufzeit $\bigO(n)$ oder mit einer angepassten Datenstruktur (siehe TODO) noch schneller ein Kreis $C_{T,e}$ bzw. dessen Orientierung $\bar{C}_{T,e}$ gefunden werden.

Im Normalfall verändern sich von einer Iteration auf die nächste nur ein Bruchteil der Kreise. Diese lassen sich jedoch schlecht herausfiltern, wenn nur die Veränderung der zulässigen Baumlösung betrachtet wird. Um die Kreise nicht jede Iteration neu berechnen zu müssen, habe ich mich entschieden, nach \cite{betreuer} nicht den Baum abzuspeichern, sondern alle Kreise. Die theoretische Komplexität bleibt etwa dieselbe,\footnotemark{} in der Praxis ist es schneller, nur die veränderten Kreise neu zu berechnen.

\footnotetext{Wenn wir Pivotalgorithmen betrachten, die in Laufzeit $\text{poly}(m-n)$ sämtliche Kreise in ihre Entscheidung einbeziehen, erhalten wir eine Komplexität von $\bigO(n\cdot(m-n) + \text{poly}(m-n))$.}

Die Klasse \lstinline|class Circle| speichert gerichtete Kreise als zwei Vektoren, die Kanten wie gehabt über Anfangsknoten, Endknoten und die \lstinline|bool isResidual| identifizieren. Außerdem speichern sie ihre jeweiligen Werte für $\delta$ in \lstinline|flow| und $c(\bar{C}_{T,e})$ in \lstinline|costPerFlow|; diese müssen jedoch vom Algorithmus gesetzt werden. Nicht explizit in der Klassendeklaration schlägt sich die Invariante nieder, dass im Verlauf des Algorithmus jeder \lstinline|Circle c| durch die erste Kante als einzige Nichtbaumkante spezifiziert wird.

\begin{lstlisting}[escapechar=ß]
//first edge of edges is not part of the underlying tree
class Circle {
public:
  //circles don't know about their graphs
  intmax_t flow = 0, costPerFlow = 0;

private:
  std::vector<std::pair<size_t, size_t>> edges;
  //std::vector<char>, because std::vector<bool> is brokenß\footnotemarkß
  std::vector<char> isResidual;
};
\end{lstlisting}

Bevor wir uns die Umsetzung anschauen, überlegen wir uns theoretisch, was die Klasse leisten muss. Dazu betrachten wir eine Iteration, in der $C_1=\bar{C}_{T,e_1}$ ausgewählt wurde. Sei $C_2=\bar{C}_{T,e_2}$ mit $e_1\neq e_2$ ein weiterer Kreis. \cref{fig:circles} stellt für die zugrundeliegenden ungerichteten Kreise $\tilde{C}_1$ und $\tilde{C}_2$ dar, wie sich $C_1$ und $C_2$ zueinander verhalten können.

\footnotetext{Für mehr Informationen siehe \cite[\textit{Item 18: Avoid using} \texttt{vector<bool>}]{STL}.}

\begin{figure}[!ht]\centering
    \includestandalone{tikz_circlesupdate}
    \caption{Beide Beispiele zeigen je zwei Kreise $\tilde{C}_i=C_{T,e_i}$. Die gestrichelten Linien gehören zum nicht vollständig dargestellten Baum $T$. Im zweiten Fall sind $\tilde{C}_1$ und $\tilde{C}_2$ nicht disjunkt.}
    \label{fig:circles}
\end{figure}

Der Algorithmus wird nun $e_1$ zum neuen Baum $T'$ hinzufügen und eine ausgehende Kante $e_1\neq e'\in E(C_{T,e_1})$ wählen, die dafür entfernt wird. Zunächst halten wir fest, dass es vor der Iteration keinen weiteren Fall als die beiden aus \cref{fig:circles} geben kann: Wären $e_1$ und $e_2$ im Schnitt der ungerichteten Kreise $E(C_{T,e_1})\cap E(C_{T,e_2})$ enthalten, dann wäre $T$ nicht kreisfrei.

Für den neuen ungerichteten Kreis gilt $C_{T',e'}=C_{T,e_1}$, zur Beibehaltung der Invariante muss \lstinline|class Circle| die gespeicherten Kanten rotieren. Für die Konstellation, dass $e_1$ im Gegensatz zu $e'$ nicht saturiert ist oder vice versa ändert sich zusätzlich die Richtung von $\bar{C}_{T',e'}$ im Vergleich zu $\bar{C}_{T,e_1}$. All dies leistet die Funktion \lstinline|Circle::rotateBy|:

\begin{lstlisting}
void Circle::rotateBy (size_t index, bool toReverse) {
  //rotate, such that the correct edge is on front
  std::rotate(edges.begin(),
              edges.begin()+index, edges.end());
  std::rotate(isResidual.begin(),
              isResidual.begin()+index, isResidual.end());
  //all but the first entry might have to be reversed
  if (toReverse) {
    //change direction of all edges
    for (size_t i = 0; i < this->length; i++) {
      std::swap(edges[i].first, edges[i].second);
      isResidual[i] = not isResidual[i];
    }
    std::reverse(edges.begin() + 1, edges.end());
    std::reverse(isResidual.begin() + 1, isResidual.end());
    }
}
\end{lstlisting}

Der einfache Fall a) aus \cref{fig:circles} ist damit abgehandelt, da $C_2$ unverändert bleibt. Dies gilt ebenfalls für Fall b), insofern $e'\notin E(C_{T,e_1})\cap E(C_{T,e_2})$. Andernfalls wird, wie \cref{fig:bypass} veranschaulicht, $C_{T',e_2}$ nun $e_1$ oder $\varphi(e_1)$ statt $e'$ enthalten.

\begin{figure}[!ht]\centering
    \includestandalone{tikz_circlesbypass}
    \caption{Ursprünglich sind $e_1$ und $e_2$ Nichtbaumkanten, die Orientierung von $e_1$ und $e'$ ist irrelevant. Der Kreis $\bar{C}_{T,e_2}$ geht von $x$ über $e_2$ nach $y$ (schwarze Kanten) zurück nach $x$ über $e'$ oder $\varphi(e')$ (rote Kanten). Im neuen Baum $T'$, der statt $e'$ die Kante $e_1$ enthält, führt $\bar{C}_{T',e_2}$ nun von $y$ nach $x$ über $e_1$ oder $\varphi(e_1)$ (blaue Kanten).}
    \label{fig:bypass}
\end{figure}

Sei $W:=E(C_{T,e_1})\cap E(C_{T,e_2})$ der Schnitt der beiden ungerichteten Kreise im alten Baum $T$. Dann gilt:
\begin{align}E(C_{T',e_2})&=\big(E(C_{T,e_2})\backslash W\big)\cup\big(E(C_{T,e_1})\backslash W\big)\nonumber\\
&=\big(E(C_{T,e_1})\cup E(C_{T,e_2})\big)\backslash W\label{eq:XOR}\end{align}

Gemäß \cref{eq:XOR} lässt sich also nach einer Iteration vom aufspannenden Baum $T$ zu $T'=T-\{e_1\}+\{e'\}$ die Kantenmenge jedes weiteren ungerichteten Kreises $C_{T,e_i}$ mit $i\neq1$  aktualisieren, indem wir eine XOR-Operation mit der alten Kantenmenge und $E(C_{T,e_1})=E(C_{T',e'})$ durchführen. Um $\bar{C}_{T',e_i}$ zu erhalten, muss der Kreis nur noch entsprechend $e_i$ gerichtet werden.

Wir werden nun die recht komplexe Funktion \lstinline|Circle::update|, die oben beschriebene Routine umsetzt, im Detail betrachten. Die Funktion bekommt \lstinline|Circle& c| übergeben, dies entspricht $\bar{C}_{T',e'}$. Zunächst suchen wir sowohl in unseren Kanten als auch in ihren umgedrehten Pendants nach $e'$. Dies ist nötig, da wir die Orientierung der Kreise zueinander nicht kennen.

Sind wir nicht fündig geworden, ist die Routine abgeschlossen, da es nichts zu tun gibt. Ansonsten erreichen wir den Zustand, dass die beiden gerichteten Kreise gegeneinander orientiert sind, also der Schnitt ihrer Kanten leer ist, indem wir gegebenenfalls \lstinline|c.rotateBy(0, true)| aufrufen. Damit ist gegeben, dass in \lstinline|Circle c| nach Notation von \cref{fig:bypass} der Weg von $y$ nach $x$ den blauen Kanten entspricht.

\begin{lstlisting}
void Circle::update(Circle& c) {
  bool iR = c.getIsResidual()[0];
  std::pair<size_t, size_t> edgeSame = c.getEdges()[0];
  std::pair<size_t, size_t> edgeReverse =
       std::make_pair(edgeSame.second, edgeSame.first);

  bool reversed = false;
  size_t index = 0;
  std::vector<std::pair<size_t, size_t>>::iterator
       it = edges.begin();
  //find edgeSame if existent
  for (index = 0; it != edges.end(); index++) {
    if (iR == isResidual[index] and
        edges[index] == edgeSame) {break;}
  it++;
  }
  //if edgeSame is in this->edges, reverse c,
  //do if-case and reverse back
  if (it != edges.end()) {
    c.rotateBy(0, true);
    reversed = true;
  }
  //else try to find edgeReverse
  else {
    it = edges.begin();
    for (index = 0; it != edges.end(); index++) {
      if (iR != isResidual[index] and 
          edges[index] == edgeReverse) {break;}
      it++;
    }
  }
\end{lstlisting}

Insofern wir $e'$ oder $\varphi(e')$ gefunden haben, zeigt der Iterator \lstinline|it| an dieser Stelle darauf. Indem wir von dieser Kante aus $x$ und $y$ suchen, finden wir heraus, welche Kanten in beiden zugrundeliegenden ungerichteten Kreisen enthalten sind. Insbesondere haben wir danach in \lstinline|size_t indexLeft| und \lstinline|indexRight| gespeichert, bis wohin bzw. ab wann wir die Kanten unseres bisherigen gerichteten Kreises beibehalten.

\begin{lstlisting}
  if (reversed or it != edges.end()) {
    std::vector<bool> toCopy (c.length, true);
    size_t indexLeft = index - 1, indexRight = index + 1;
    
    //take out all edges existent in both circles
    toCopy[0] = false; //first edge not even in tree
    //left from first edge
    for (; indexLeft > 0; indexLeft--) {
      //circles are in opposite directions,
      //so reverse the edge for existence check
      std::pair<size_t, size_t> checkEdge =
      std::make_pair(
           c.getEdges()[index - indexLeft].second,
           c.getEdges()[index - indexLeft].first);
      iR = c.getIsResidual()[index - indexLeft];
      if (iR != isResidual[indexLeft] 
          and edges[indexLeft] == checkEdge)
      {toCopy[index - indexLeft] = false;}
      else {break;}
    }
    //right from first edge
    for (; indexRight < this->length; indexRight++) {
      //circles are in opposite directions,
      //so reverse the edge for existence check
      std::pair<size_t, size_t> checkEdge =
      std::make_pair(
           c.getEdges()[c.length+index-indexRight].second,
           c.getEdges()[c.length+index-indexRight].first);
      iR = c.getIsResidual()[c.length+index-indexRight];
      if (iR != isResidual[indexRight]
          and edges[indexRight] == checkEdge)
      {toCopy[c.length + index - indexRight] = false;}
      else {break;}
    }
\end{lstlisting}

Anschließend konstruiere ich der Einfachheit halber die beiden Vektoren \lstinline|edges| und \lstinline|isResidual| komplett neu. Der neue Kreis besteht dann wie in \cref{fig:bypass} aus dem Weg von $e_2$ nach $y$, gefolgt von den neuen, im Bild blauen Kanten von $y$ nach $x$ und dem Schluss von $x$ zu $e_2$. Falls wir \lstinline|Circle c| umgedreht haben, machen wir dies als letztes noch rückgängig.
    
\begin{lstlisting}
    //construct new circle
    std::vector<std::pair<size_t, size_t>> edgesNew;
    std::vector<char> isResidualNew;
    
    //first part of old circle
    for (size_t i = 0; i <= indexLeft; i++) {
      edgesNew.push_back(this->edges[i]);
      isResidualNew.push_back(this->isResidual[i]);
    }
    //bypass
    for (size_t i = 0; i < c.length; i++) {
      if (toCopy[i]) {
        edgesNew.push_back(c.getEdges()[i]);
        isResidualNew.push_back(c.getIsResidual()[i]);
      }
    }
    //last part of old circle
    for (size_t i = indexRight; i < this->length; i++) {
      edgesNew.push_back(this->edges[i]);
      isResidualNew.push_back(this->isResidual[i]);
    }
    
    this->edges = edgesNew;
    this->isResidual = isResidualNew;
  }

  //reverse back
  if(reversed) {c.rotateBy(0, true);}
}
\end{lstlisting}

Mit diesen Datenstrukturen als Grundlage können wir nun den Algorithmus implementieren.

\section{Netzwerk-Simplex-Algorithmus}
Die Klasse \lstinline|class Algorithm| besteht nur aus einem Konstruktor und der Funktion \lstinline|Algorithm::solution|, die das Min-Cost-Flow-Problem entweder mit LC- oder HC-Initialisierung (siehe \cref{ch:init} für die Funktionsweise und \cref{ch:HCLC} für die Umsetzung) löst. Ein \lstinline|Algorithm a| wird mit einer konkreten Pivotfunktion erstellt, die aus einem \lstinline|std::vector<Circle>| einen Kreis auswählt und dessen Index zurückgibt.
\begin{lstlisting}
class Algorithm {
public:
  Algorithm (Network& _n, size_t (*_pivot)
                          (const std::vector<Circle>&))
    : n(_n), pivot(_pivot) {};

  //returns 0 if n cannot be solved
  //false for high cost edge initialization
  bool solution (bool modus = true);
}
\end{lstlisting}

Die Klasse speichert intern das Netzwerk, den Pivotalgorithmus, die Kreise, die \lstinline|id| des künstlichen Knoten sowie die Anzahl an Iterationen ab. Zusätzlich gibt es eine Funktion zur anfänglichen Erzeugung aller Kreise per Tiefensuche.

\begin{lstlisting}
class Algorithm {
private:
  intmax_t iterations = 0;
  Network& n;
  size_t (*pivot)(const std::vector<Circle>&);
  size_t artNode;
  std::vector<Circle> circles;

  //false if no optimization was possible
  bool optimize();
  //takes tree and creates circles;
  void createCircles(std::vector<Node> tree);
};
\end{lstlisting}

Der Kern der Klasse ist die Funktion \lstinline|Algorithm::optimize|, die eine Iteration durchführt oder \lstinline|false| zurückgibt, falls dies nicht mehr möglich und damit Phase 2 beendet ist. Nachdem Phase 1 abgeschlossen ist, wird dementsprechend die Schleife \lstinline|while (optimize());| ausgeführt.
 
Die Funktion aktualisiert zunächst die Werte \lstinline|flow| und \lstinline|costPerFlow| für die Kreise, insofern notwendig. Dann wählt sie durch den Pivotalgorithmus einen Kreis \lstinline|Circle c| für die nächste Iteration aus oder gibt \lstinline|false| zurück, falls es keinen negativen Kreis mehr gibt. \lstinline|c| wird um \lstinline|c.flow| augmentiert und gemäß \cref{ch:deg} wird die Kante $e'$ gewählt, die den Baum verlässt. Zum Schluss wenden wir \lstinline|Circle::update| für alle anderen Kreise an. Mit unseren bisher vorgestellten Datenstrukturen und Codeschnipseln ist alles bis auf die Wahl von $e'$ leicht implementierbar. Das nächste Kapitel beschäftigt sich mit der Umsetzung dieses Teilproblems.

\subsection{Stark zulässige Baumlösungen}
TODO: Fun Fact – um stark zulässige Baumlösungen zu erhalten, muss ich vermutlich doch den Baum speichern. Na ja.

\subsection{Pivotalgorithmen}
Die Pivotalgorithmen werden in meiner Implementierung in \lstinline|PivotAlgorithms.h| gesammelt. Ein Pivotalgorithmus gibt einen ungültigen Index zurück, falls es keinen negativen Kreis gibt.

\begin{lstlisting}
//return index of chosen circle or circles.size() if
//no circle can be chosen

//returns a random circle with negative cost
size_t pivotRandom(const std::vector<Circle>& circles);
//returns most negative circle
size_t pivotMaxVal(const std::vector<Circle>& circles);
//returns max |flow*costperflow|
size_t pivotMaxRev(const std::vector<Circle>& circles);
\end{lstlisting}

Beispielhaft schauen wir uns noch die Umsetzung von \lstinline|pivotMaxVal| an:

\begin{lstlisting}
size_t pivotMaxVal(const std::vector<Circle>& circles) {
  intmax_t mini = 0;
  size_t index = circles.size();
  for (size_t i = 0; i < circles.size(); i++) {
    intmax_t value = circles[i].costPerFlow;
    if (value < mini) {mini = value; index = i;}
  }
  return index;
}
\end{lstlisting}

\subsection{Initialisierung}\label{ch:HCLC}
Wir haben nun alle essentiellen Bestandteil der Implementierung von Phase 2 des Netzwerk-Simplex-Algorithmus gesehen. Zum vollständigen Algorithmus fehlt uns nur noch Phase 1, wie in \cref{ch:init} beschrieben. Zu Beginn legen wir je nach Modus die Kosten der künstlichen Kanten fest.

\begin{lstlisting}[escapechar=ß]
bool Algorithm::solution (bool modus) {
  iterations = 0;
  //high cost edge
  intmax_t maxCost = 0;
  if (modus == false) {
    for (autoß\footnotemarkß edge : n.getEdges()) {
      if (edge.second.cost > maxCost) {
        maxCost = edge.second.cost;
      }
    }
  }
  //toggle all edges
  else {n.toggleCost();}
  maxCost = maxCost*n.getNoOfNodes() + 1;
\end{lstlisting}

\footnotetext{\lstinline|const std::pair<const std::tuple<size_t, size_t, bool>, Edge>&|}

Nun legen wir den künstlichen Knoten und die künstlichen Kanten an, hier beispielhaft für die Quellen:
\begin{lstlisting}
  artNode = n.addNode(0);
  Node artNode_Tree = Node(artNode, 0);

  for (size_t source : n.sources) {
    n.addEdge(Edge(source, artNode, maxCost,
                   n.sumSource + 1));
  }
\end{lstlisting}

Die Funktion \lstinline|bool Algorithm::solution| gibt sofort \lstinline|false| zurück, falls der übergebene Graph $G$ die Gleichung $\sum_{v\in V(G)} b(v)=0$ nicht erfüllt. Somit können wir bei der Erzeugung des initialen Flusses $f_0$ Wege von Quellen zu Senken über den künstlichen Knoten augmentieren, bis der b-Wert eines der beiden Endknoten erfüllt ist, und dann mit der nächsten Quelle bzw. Senke iterieren.

\begin{lstlisting}
  //get that flow started
  std::vector<size_t>::iterator iSo = sources.begin(),
                                iSi = sinks.begin();
  while (iSo != sources.end()) {
    if (std::abs(n.getNodes().find(*iSo)->second.b_value)
     <= std::abs(n.getNodes().find(*iSi)->second.b_value)){
      std::vector<size_t> path = {*iSo, artNode, *iSi};
      n.addFlow(path,
        std::abs(n.getNodes().find(*iSo)->second.b_value));
      iSo++;
    }
    else {
      std::vector<size_t> path = {*iSo, artNode, *iSi};
      n.addFlow(path,
        std::abs(n.getNodes().find(*iSi)->second.b_value));
      iSi++;
    }
  }
\end{lstlisting}

Die Initialisierung ist an dieser Stelle abgeschlossen, nun können wir die Instanz lösen.

\begin{lstlisting}
  //complete tree and create circles
  createCircles(tree);

  //solve with artificial node
  while (optimize());
\end{lstlisting}

Ist \lstinline|modus = false|, dann befinden wir uns in HC und sind an dieser Stelle fertig. Kann der künstliche Knoten gelöscht werden, ist die Instanz lösbar und die Lösung im Netzwerk abgespeichert, andernfalls gibt \lstinline|solution| den Wert \lstinline|false| zurück.

Für LC wissen wir zunächst nur, ob die Instanz lösbar ist oder nicht. Wenn sie lösbar ist, können wir den künstlichen Knoten löschen, über \lstinline|Network::toggleCost| die originalen Kosten wiederherstellen und mit \lstinline|optimize| eine optimale Lösung finden.

Um die dafür notwendige stark zulässige Baumlösung auf dem Netzwerk ohne künstlichen Knoten zu finden, gibt es zwei Ansätze: Ein Weg ist es, einfach auf dem Graphen einen neuen Baum $T'$ zu bestimmen, der mit dem aktuellen Fluss $f$ eine zulässige Baumlösung $(T',f)$ bildet. In meiner Umsetzung führen wir dagegen vor der Löschung des künstlichen Knoten degenerierte Iterationen durch, bis dieser ein Blatt ist. Dies schien mir der kohärente Ansatz zu sein.

Dafür gehen wir über den Vektor \lstinline|circles| und suchen nach Kreisen, in denen zwei Baumkanten mit dem künstlichen Knoten benachbart sind. Sei $e$ diejenige der beiden Kanten, die den Baum verlässt. $e$ kann nur in schon betrachteten Kreisen enthalten sein, wenn deren identifizierende Kante mit dem künstlichen Knoten benachbart ist; diese Kreise werden jedoch später gelöscht. Also genügt es, nur Kreise über \lstinline|update| zu aktualisieren, die noch nicht betrachtet wurden, womit folgende Subroutine terminiert.

\begin{lstlisting}
  for (std::vector<Circle>::iterator it = circles.begin();
       it != circles.end(); it++) {
  //find edge over artificial node if existent
  size_t i = 1;
  for (; i < it->size(); i++) {
    if (it->getEdges()[i].first == artificialNode or
        it->getEdges()[i].second == artificialNode)
      {break;}
  }
  if (i < it->size()) {
    //since n is feasible, one of the both cases occures

    //not residual <==> flow == 0
    //new first edge, same direction
    if (not it->getIsResidual()[i])
      {it->rotateBy(i, false);}
    //new first edge, but reversed direction
    else {it->rotateBy(i, true);}

    //circles before this one don't need an update 
    for (std::vector<Circle>::iterator otherCircle = it+1;
         otherCircle != circles.end(); otherCircle++)
      {otherCircle->update(*it);}
    iterations++;
  }
\end{lstlisting}

Zuletzt werden der künstliche Knoten und wie angedeutet die $|V(G)|-1$ Kreise gelöscht, die am künstlichen Knoten beginnen, bevor die finale Lösung berechnet wird. Ich benutze an dieser Stelle eine Lambdafunktion, um die entsprechenden Kreise herauszufiltern.

\begin{lstlisting}
  //now remove all circles beginning at the artificial node
  circles.erase(std::remove_if(circles.begin(), 
                               circles.end(),
    [this](const Circle& c)
    {return c.getEdges()[0].first == artificialNode or
            c.getEdges()[0].second == artificialNode;}
    ), circles.end());

  while (optimize());
  return true;
}
\end{lstlisting} 
\chapter{Exponentielle Instanzen aus der Literatur}\label{ch:lit}

\chapter{Experimentelle Ergebnisse}\label{ch:erg}
Meh.

\chapter{Ausblick}
La la la.

\backmatter
\bibliography{bibliography}{}
\bibliographystyle{ieeetr}

\end{document}