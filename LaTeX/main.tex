\documentclass[a4paper,twoside]{report}
\author{Max~Kanold}
\title{Experimentelle Ergebnisse zum Network-Simplex-Algorithmus}

\usepackage[utf8]{inputenc}
\usepackage[T1]{fontenc}

\usepackage{mathtools}

\usepackage[hidelinks]{hyperref}
% or colorlinks

\usepackage{amssymb,amsthm}

\theoremstyle{plain}
\newtheorem{thm}{Theorem}

\theoremstyle{definition}
\newtheorem{defn}[thm]{Definition}

\newtheorem*{anm}{Anmerkung}

\usepackage[shortlabels]{enumitem}

\usepackage[ngerman]{babel}
\usepackage{csquotes}
\usepackage{anyfontsize}
\usepackage[babel]{microtype}


\begin{document}
\maketitle
\tableofcontents

\newpage
\chapter{Einführung}
Bla. Zum Beispiel in Kapitel \ref{prog} habe ich programmiert.

\newpage
\chapter{Network-Simplex-Algorithmus}
Das Simplex-Verfahren, zu welchem eine Einführung in \cite{NSAbook} gefunden werden kann, löst Lineare Programme in der Praxis sehr schnell, obwohl die Worst-Case-Laufzeit nicht polynomiell ist. Jedes Netzwerkproblem lässt sich als Lineares Programm darstellen und somit durch das Simplex-Verfahren lösen, durch die konkrete Struktur solcher Probleme genügt jedoch der vereinfachte Network-Simplex-Algorithmus. Auch für diesen gibt es exponentielle Instanzen (siehe \cite{Exponential}), in der Praxis wird er trotzdem vielfach verwendet.

\section{Min-Cost-Flow-Problem}
\begin{defn}Ein \textbf{Netzwerk} ist ein Tupel $(G,b,c,u)$, wobei $G = (V,E)$ ein gerichteter Graph, $b : V\rightarrow\mathbb{R}$ eine b-Wert-Funktion, $c : E\rightarrow\mathbb{R}$ eine Kostenfunktion und $u : E\rightarrow\mathbb{R}_{\geq 0}$ eine Kapazitätsfunktion seien.\end{defn}
\begin{anm}Knoten mit positiven b-Wert werden als Quellen, solche mit negativen als Senken bezeichnet.\\
Ein ungerichteter Graph kann durch das Verwandeln jeder Kante $\{v,w\}$ in zwei Kanten $(v,w)$ und $(w,v)$ zu einem gerichteten modifiziert werden.\end{anm}

\begin{defn}Ein \textbf{maximaler Fluss} auf einem Netzwerk $(G=(V,E),b,c,u)$ ist eine Abbildung $f : E\rightarrow\mathbb{R}_{\geq 0}$, die folgende Eigenschaften erfüllt:
\begin{enumerate}[(i)]
\item $\forall e\in E : f(e)\leq u(e) $
\item $\forall v\in V : \sum_{(w,v)\in E} f((w,v)) - \sum_{(v,w)\in E} f((v,w)) + b(v) = 0$
\end{enumerate}
Der \textbf{Wert} von $f$ ist
$v(f) = \frac{1}{2}\cdot\sum_{v\in V} |b(v)|$,\\
die \textbf{Kosten} von $f$ sind
$c(f) = \sum_{e\in E} f(e)\cdot c(e)$.
\end{defn}

Beim \emph{Min-Cost-Flow-Problem} wird unter allen maximalen Flüssen einer mit minimalen Kosten gesucht. Sind die Kapazitäten unbeschränkt, so wird es als \emph{Transportproblem} bezeichnet.

Für diese Bachelorarbeit wurde angenommen, dass $u$ und $c$ auf $\mathbb{N}$ sowie $b$ auf $\mathbb{Z}$ abbildet, um Gleitkommazahlungenauigkeit zu vermeiden. Durch eine entsprechende Skalierung des Problems können die Funktionen nach $\mathbb{R}_{\geq 0}$ bzw. $\mathbb{R}$ hinreichend genug angenähert werden. Zusätzlich wird davon ausgegangen, dass $\sum_{v\in V(G)} b(v) = 0$ ist, Angebot und Nachfrage also ausgeglichen sind. Des Weiteren ist in der konkreten Implementierung $E$ keine Multimenge; es sind keine parallelen Kanten vorgesehen.

\section{Algorithmus}
\cite[Dantzig, 1951]{erf1} und \cite[Orden, 1956]{erf2} vereinfachten das Simplex-Verfahren zum Netzwerk-Simplex-Algorithmus; die folgende Beschreibung orientiert sich zuerst an \cite[S. 291\,ff.]{NSAbook} zur Lösung des Transportproblems, danach wird der Algorithmus anhand von TODO auf den allgemeinen, durch Kapazitäten beschränkten Fall erweitert.

\begin{defn}Ein \textbf{Baum} $T$ ist ein ungerichteter, zusammenhängender und kreisfreier Graph.\\
Ein Teilgraph $T=(V',E')$ eines ungerichteten Graphen $G=(V,E)$ heißt \textbf{aufspannender Baum}, wenn T ein Baum und $V'=V$ ist.\end{defn}
\begin{anm}Sprechen wir bei einem gerichteten Graphen $G$ über einen aufspannenden Baum, so bezieht sich das stets auf einen aufspannenden Baum des $G$ zugrundeliegenden ungerichteten Graphen.\end{anm}




\section{Implementation} \label{prog}
Hier beginnt mein schönes Werk \ldots

\subsection{Spezielle Konstrukte}
\ldots und hier endet es.

\subsubsection{Die Klasse Circle}
Kreise halt.\cite{NSAbook}

\subsubsection{Der Rest halt}
Kleinkram.

\newpage
\chapter{Experimentelle Ergebnisse}
Alle scheiße.

\newpage
\chapter{Ausblick}
La la la.

\bibliography{bibliography}{}
\bibliographystyle{ieeetr}

\end{document}