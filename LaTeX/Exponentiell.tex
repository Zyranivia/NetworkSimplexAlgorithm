\chapter{Exponentielle Instanzen nach Zadeh}\label{ch:lit}
Zadeh veröffentlichte 1973 in \cite{Exponential} eine Familie von exponentiellen Instanzen des Transportproblems für diverse verbreitete Algorithmen. Dabei wählte er im Falle des Netzwerk-Simplex-Algorithmus eine initiale stark zulässige Baumlösung, die an den Baum der High-Cost-Initialisierung erinnert. Damit können wir rekursiv Netzwerke $N_j$ für $j\geq 3$ mit $2j$ Knoten definieren, für die der programmierte Algorithmus mit HC und MaxVal $2^j + 2^{j-1} - 2$ Iterationen benötigt. \cref{fig:N3} zeigt den Rekursionsanfang $N_3$.
\begin{figure}[!ht]\centering
    \includestandalone{tikz_N3}
    \caption{Grafik angelehnt an \cite[S. 261]{Exponential}. Alle Kapazitäten sind unendlich, die Kosten wie angegeben. Die sechs mittleren Knoten bilden das Netzwerk $N_3$, die gestrichelten Linien stellen den initialen Baum dar, wobei die beiden Knoten $a$ identisch sein sollen. Die rekursive Erzeugung des Netzwerkes $N_j$ wird angedeutet.}
    \label{fig:N3}
\end{figure}

Es folgt zunächst eine beispielhafte Implementierung des Netzwerkes $N_3$, bevor wir die Rekursion betrachten und umsetzen:

\begin{lstlisting}
Network n3 = Network(0);
std::vector<intmax_t> b_value = {1, -2, 3, -2, 5, -5};
for (intmax_t b : b_value) {n3.addNode(b);}

intmax_t u = std::pow(2, 30); //"infinite" capacity
std::vector<size_t> from = {0,1,0,0,2,2,4,4};
std::vector<size_t> to   = {1,0,3,5,1,5,1,3};
std::vector<intmax_t> co = {0,0,1,3,1,3,3,3};
std::vector<intmax_t> ca = {u,u,u,u,u,u,u,u};

for (size_t i = 0; i < from.size(); i++) {
  n3.addEdge(Edge(from[i], to[i], co[i], ca[i]))
}
\end{lstlisting}

Wie \cref{fig:N3} andeutet, entsteht das Netzwerk $N_j$, indem wir zu $N_{j-1}$ zwei neue Knoten $q_j$ und $s_j$ mit b-Wert $b_j:=2^{j-1} + 2^{j-3}$ und $-b_j$ hinzufügen. Wir fixieren $c_j:=2^{j-1}-1$ als Kosten. Die Quelle $q_j$ verbinden wir mit jeder Senke $s_i\neq s_j$ über eine Kante mit Kosten $c((q_j,s_i))=c_j$; die Senke $s_j$ ist von jeder Quelle $q_i\neq q_j$ aus mit Kosten $c((q_i,s_j))=c_j$ erreichbar.

Zadeh legt in \cite[S. 263\,f.]{Exponential} dar, dass MaxVal\=/HC bei diesen Netzwerken in jeder nicht-degenerierten Iteration einen Kreis, der über den künstlichen Knoten verläuft, um $\delta=1$ augmentiert. Da nach Konstruktion der Durchfluss von $N_j$ exponentiell steigt, ist die Anzahl an Iterationen ebenfalls exponentiell.

Wenn wir bereits ein \lstinline|Network n| haben, das $N_{j-1}$ entspricht, führt uns folgender Code zu $N_{j}$. Es wird davon ausgegangen, dass für eine Quelle $q_i$ \lstinline|id = 2*i| und für eine Senke $s_j$ \lstinline|id = 2*i + 1| gilt. Außerdem sollte \lstinline|intmax_t u| hinreichend groß sein.

\begin{lstlisting}
size_t j = n.getNoOfNodes()/2 + 1;
intmax_t b_value = std::pow(2, j-1) + std::pow(2, j-3);
size_t q_j = n.addNode(b_value), s_j = n.addNode(-b_value);

intmax_t c_j = std::pow(2, j-1) - 1;
for (size_t i = 0; i < j-1; i++) {
  test2.addEdge(Edge(q_j, 2*i + 1, c_j, u));
  test2.addEdge(Edge(2*i, s_j, c_j, u));
}
\end{lstlisting}

In \cref{fig:plot_exp} ist die Laufzeit aller sechs eingeführten Varianten des Netzwerk-Simplex-Algorithmus auf den Netzwerken $N_3$ bis $N_{10}$ dargestellt. MaxVal\=/HC steigt exponentiell, während derselbe Pivotalgorithmus mit LC linear ist. Die randomisierten Versionen scheinen stärker als linear zu wachsen, jedoch nicht exponentiell.

Damit lässt sich vermuten, dass Familien von exponentiellen Instanzen nicht nur einzeln pro Pivotalgorithmus, sondern zusätzlich differenziert nach der Initialisierung zu suchen sind. Im nächsten Kapitel beschäftigen wir uns mit algorithmischen Ansätzen für diese Suche. 

\begin{figure}[!ht]\centering
    \includestandalone{plot_exp}
    \caption{Anzahl der Iterationen der sechs eingeführten Algorithmusvarianten auf den Netzwerken $N_i$. Die Laufzeit von MaxVal\=/HC ist exponentiell, die der drei anderen deterministischen Varianten ist linear. Für den randomisierten Ansatz wurde der Schnitt von $|V(G_i)|$ Durchläufen genommen.}
    \label{fig:plot_exp}
\end{figure}
